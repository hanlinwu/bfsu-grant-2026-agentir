\section{课题负责人近五年来已有的相关研究成果}

课题负责人近五年持续围绕计算机视觉与遥感智能处理开展研究,研究方向主要包括:图像恢复、连续比例因子超分辨率、视觉语言多模态学习、基于扩散模型的生成式方法、遥感变化分析等。

根据个人主页公开成果,近年已形成较为系统的研究积累。在论文方面,相关成果发表于IEEE Transactions on Geoscience and Remote Sensing、IEEE Geoscience and Remote Sensing Letters、Remote Sensing、ICASSP、IGARSS等期刊与会议,研究主题覆盖:盲超分辨率重建、连续比例因子超分辨率、跨模态检索、变化描述与变化问答等。

代表性成果可归纳为三类,具体如下。

一、\textbf{连续比例因子超分辨率建模}  
问题:固定比例模型训练/存储成本高,且对非整数倍率泛化不足。  
方法:构建连续尺度感知建模框架,引入跨尺度特征耦合与动态上采样机制。  
贡献:实现单模型覆盖连续倍率重建并兼顾质量与效率。  
结果:形成高水平论文成果并沉淀可复用模型组件。  
与本项目关联:为“区域工具参数连续优化”提供直接方法学基础。

二、\textbf{复杂退化过程下的生成式恢复}  
问题:真实场景退化分布复杂,传统单一退化假设鲁棒性不足。  
方法:结合生成式建模与退化过程表征,提升模型对复杂退化的适应能力。  
贡献:增强复杂场景中的恢复稳定性与视觉一致性。  
结果:形成复杂退化恢复方向的连续研究成果。  
与本项目关联:为“空间退化图驱动的策略规划”提供可迁移技术储备。

三、\textbf{遥感多模态理解与指令驱动分析}  
问题:仅依赖视觉特征难以支撑高层语义决策。  
方法:开展跨模态检索、变化描述与变化问答研究,探索指令驱动视觉分析流程。  
贡献:形成视觉—语言联合推理能力与任务编排经验。  
结果:建立多模态任务原型与评测流程。  
与本项目关联:为“多智能体协同推理与反馈”提供实现基础。

科研成果社会评价方面,相关工作已在领域主流期刊/会议发表并被持续关注,研究路线具有明确延续性。上述积累与本课题拟解决的“复杂退化场景下的智能恢复决策”问题具有直接衔接关系。

