\section{前期准备工作}

本节从“已完成事项 + 产出物 + 可验证状态”三个维度说明前期准备,并与本课题研究内容1/2/3建立对应关系。

\subsection{理论与技术储备}

(1)已形成图像恢复与超分辨率方法学储备(对应研究内容1/2)。产出物:模型设计与评测方案。可验证状态:已有多篇相关成果与可复现流程。

(2)已形成多模态遥感理解储备(对应研究内容2)。产出物:跨模态任务原型与分析脚本。可验证状态:已完成多类任务实验闭环。

(3)已形成生成式建模储备(对应研究内容1/3)。产出物:生成式建模实验管线。可验证状态:已具备复杂退化建模与对比验证基础。

\subsection{数据与实验环境构建}

(1)已建立遥感图像恢复与多模态任务实验环境。产出物:统一训练/验证/消融流程。可验证状态:可稳定复现主流方法。

(2)已完成多类任务基线复现与评测流程搭建。产出物:基线模型与对比评测脚本。可验证状态:可生成标准化评测结果,为后续方案提供可靠对照。

(3)已具备公开预训练模型与开源代码接入能力。产出物:可复用模块接入模板。可验证状态:可快速开展新模块原型验证。

(4)阶段完成度:目前已完成基础环境与基线体系建设,正在推进面向多智能体决策的专项实验管线建设。

\subsection{关键技术预研}

(1)预研问题:连续比例因子恢复中的效率与质量平衡。技术路径:轻量化模型与动态参数策略。初步结果:已验证该路径可行。后续计划:纳入区域级参数调优机制。

(2)预研问题:复杂场景下指令驱动决策稳定性。技术路径:视觉语言联合分析与反馈闭环。初步结果:已验证指令驱动分析有效。后续计划:扩展至多智能体交互流程。

(3)预研问题:复杂退化建模与部署效率矛盾。技术路径:生成式建模 + 轨迹蒸馏。初步结果:已验证潜空间建模鲁棒性。后续计划:推进轻量模型蒸馏与边缘部署测试。

(4)风险识别与改进策略:针对预研中可能出现的训练不稳定、跨场景泛化不足等问题,已制定困难样本增强、分阶段训练与多指标联合评估策略。

