\section{创新之处}

本项目创新点按"理论创新—方法创新—技术创新"组织如下:

\textbf{一是理论创新}:提出"空间退化状态驱动的恢复决策"框架,将图像恢复问题从全图单步映射扩展为区域状态感知下的多步决策问题。

传统图像恢复方法将整幅图像视为单一退化实体,采用统一的恢复函数$\hat{\mathbf{x}} = f(\mathbf{y})$进行端到端映射。这一范式忽视了真实场景中退化的空间异质性——同一张图像的不同区域可能同时受到类型迥异、程度各异的退化影响。本项目提出空间退化图$\mathcal{M} = \{(R_i, \mathbf{d}_i, \mathbf{s}_i)\}$作为决策先验,将恢复问题重新定义为状态条件策略函数$\hat{\mathbf{x}}_{R_i} = \pi(\mathbf{y}_{R_i} | \mathcal{M})$,使恢复策略能够根据区域退化状态动态调整。相对已有方法的增量边界在于:由"统一恢复函数"转向"状态条件策略函数",为复杂退化场景下的精细化恢复提供理论支撑。

\textbf{二是方法创新}:提出全局调度器与区域专家协同的层级多智能体机制,实现工具链选择与连续参数优化一体化。

现有基于智能体的恢复方法(如AgenticIR、MAIR)虽引入动态决策能力,但均采用整图级统一规划,未能利用空间分布信息。本项目提出"全局调度器 + 区域专家"的两层协同架构:全局层依据退化先验(场景退化$\to$成像退化$\to$压缩退化的逆序处理原则)分配处理优先级;区域层执行工具链选择与连续参数优化,并通过质量反馈迭代更新。该机制突破了离散工具选择与连续参数调节的分离局限,实现了"离散路由 + 连续调参"的联合优化。相对已有方法的增量边界在于:从全图统一决策扩展到区域级差异化决策,从单一任务选择扩展到工具链序列规划与参数自适应。

\textbf{三是技术创新}:提出基于轨迹蒸馏的轻量路由部署方案,将高质量多步推理能力压缩为可实时执行的轻量模型。

基于LLM/VLM的智能体系统虽具备强大的规划能力,但在线推理成本高、部署门槛高,难以满足边缘实时应用需求。本项目提出离线轨迹收集与蒸馏训练相结合的技术路线:首先利用多智能体系统在合成数据上离线运行,收集"输入图像$\to$最优决策"的轨迹数据;然后训练轻量级路由网络直接预测区域分割、退化类型、工具调度与参数配置,实现端到端推理。部署时无需依赖大模型在线调用,仅需轻量网络与恢复工具库即可完成实时恢复。相对已有方法的增量边界在于:从"依赖大模型在线推理"转向"离线蒸馏后边缘高效推理",在保持恢复质量的同时显著降低部署时延与算力开销。

可验证创新产出包括:可复现实验协议、系统原型与关键模块实现、阶段论文与专利成果。
