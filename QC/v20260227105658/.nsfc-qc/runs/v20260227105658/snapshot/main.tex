% !TEX program = xelatex
\documentclass[a4paper]{article}

% ===== 中文支持 =====
\usepackage[UTF8, fontset=none]{ctex}
\setCJKmainfont[BoldFont=SimHei, ItalicFont=KaiTi]{SimSun}
\setCJKsansfont{SimHei}
\setCJKmonofont{FangSong}
\setmainfont{Times New Roman}

% ===== 页面设置(与 template.docx 一致) =====
\usepackage[
  a4paper,
  left=2.86cm,
  right=3.17cm,
  top=2.54cm,
  bottom=2.54cm
]{geometry}

% ===== 宏包 =====
\usepackage{mdframed}
\usepackage{graphicx}
\usepackage{float}
\usepackage{amsmath,amssymb}
\usepackage[hidelinks]{hyperref}
\usepackage{xcolor}

% ===== 参考文献管理 =====
\usepackage[backend=biber,style=gb7714-2015,sorting=none,maxbibnames=3,minbibnames=1]{biblatex}
\addbibresource{refs.bib}

% ===== 默认字号:五号(10.5pt),正文 1.2 倍行距 =====
\renewcommand{\normalsize}{\fontsize{10.5pt}{14pt}\selectfont}
\normalsize

% ===== 段落设置 =====
\setlength{\parindent}{0pt}
\setlength{\parskip}{0pt}
\pagestyle{empty}

% ===== 一级标题样式:黑体小四(12pt) =====
\usepackage{titlesec}
\titleformat{\section}[block]
  {\fontsize{12pt}{20pt}\selectfont\sffamily}  % 黑体小四
  {\thesection.~}                               % 编号格式
  {0pt}                                         % 编号与标题间距
  {}
\titlespacing*{\section}{0pt}{6pt}{3pt}         % 左缩进、段前、段后
\renewcommand{\thesection}{\arabic{section}}    % 用阿拉伯数字编号

% ===== 二级标题样式:黑体五号(10.5pt) =====
\titleformat{\subsection}[block]
  {\fontsize{10.5pt}{20pt}\selectfont\sffamily}   % 黑体五号
  {\thesubsection.~}                               % 编号格式
  {0pt}
  {}
\titlespacing*{\subsection}{0pt}{4pt}{2pt}         % 左缩进、段前、段后
\renewcommand{\thesubsection}{\thesection.\arabic{subsection}}

% ===== 可分页边框内容区 =====
\newmdenv[
  skipabove=0pt,
  skipbelow=0pt,
  everyline=true,
  innerleftmargin=5pt,
  innerrightmargin=5pt,
  innertopmargin=5pt,
  innerbottommargin=5pt,
  linewidth=0.4pt,
  middlelinewidth=0.4pt,
  linecolor=black,
  middlelinecolor=black,
  backgroundcolor=white,
  roundcorner=0pt
]{proposalbox}

\newmdenv[
  skipabove=0pt,
  skipbelow=0pt,
  everyline=true,
  innerleftmargin=5pt,
  innerrightmargin=5pt,
  innertopmargin=5pt,
  innerbottommargin=5pt,
  linewidth=0.75pt,
  middlelinewidth=0.75pt,
  linecolor=black,
  middlelinecolor=black,
  backgroundcolor=white,
  roundcorner=0pt
]{proposalboxthick}

% ===== 正文开始 =====
\begin{document}

% ==============================================
%  第一部分:项目设计论证
% ==============================================

% 标题:黑体 四号(14pt),双倍行距
\noindent{\fontsize{14pt}{28pt}\selectfont\sffamily 三、项目设计论证\par}%
\vspace{1em}
\nointerlineskip

% --- 可分页边框区:说明 + 填写区域 ---
\begin{proposalboxthick}

{\fontsize{10.5pt}{10.5pt}\selectfont\bfseries
本表参照以下提纲撰写,要求逻辑清晰,主题突出,层次分明,内容翔实,排版清晰,6000字以内。\par}%
\vspace{4pt}%
{\fontsize{10.5pt}{12.5pt}\selectfont
\hspace*{0.59cm}1.[\textbf{选题依据}]\hspace{0.5em}国内外相关研究的学术史梳理及研究动态;本课题相对于已有研究的独到学术价值和应用价值等。%
\par
\hspace*{0.59cm}2.[\textbf{研究内容}]\hspace{0.5em}本课题的研究对象、总体框架、重点难点、主要目标等。%
\par
\hspace*{0.59cm}3.[\textbf{思路方法}]\hspace{0.5em}本课题研究的基本思路、具体研究方法、研究计划及其可行性等。%
\par
\hspace*{0.59cm}4.[\textbf{创新之处}]\hspace{0.5em}在学术思想、学术观点、研究方法等方面的特色和创新。%
\par
\hspace*{0.59cm}5.[\textbf{预期成果}]\hspace{0.5em}成果形式、使用去向及预期社会效益等。%
\par
\hspace*{0.59cm}6.[\textbf{参考文献}]\hspace{0.5em}开展本课题研究的主要中外参考文献。%
\par\vspace{6pt}%
}%


% ========================================
% 在此填写内容(以下为示例结构,可自行修改)
% ========================================

\setlength{\parindent}{2em}%

\section{选题依据}

本节聚焦“为什么必须做、现有方法为何不足、本项目价值何在”三个核心问题,形成从问题提出到目标牵引的完整论证链条。
本节结论将直接支撑第2节的研究目标设定与总体框架设计。

\subsection{研究意义}

图像恢复旨在从受损的观测图像中重建高质量的清晰图像,是计算机视觉领域的基础性研究课题。在安防监控、自动驾驶、医学成像、遥感观测等实际应用中,图像在采集与传输过程中不可避免地受到噪声、模糊、雾霾、雨线、低光照、压缩伪影等多种退化因素的影响,严重制约了后续视觉分析任务的精度与可靠性。因此,研究高效、智能的图像恢复方法,对提升视觉信息质量和推动计算机视觉技术的实用化部署具有重要意义。

当前,基于深度学习的图像恢复算法取得了显著进展,但仍面临以下挑战:

一、传统方法\cite{swinir,restormer}针对单一退化类型设计专用模型,面对真实场景中多种退化共存的复杂情况时难以有效应对;

二、全能型(All-in-One)方法\cite{airnet,promptir,instructir}虽尝试用单一模型处理多种退化,但受限于训练阶段所覆盖的退化类型,在面对未知混合退化时性能显著下降;

三、现有方法普遍采用静态推理范式,无法根据图像的实际退化情况动态选择和组合恢复策略。

在本课题预研复现实验中,上述问题在复杂混合退化场景下会造成明显性能损失:与单一退化测试相比,代表性模型在PSNR上通常下降约1.5--3.0dB,在视觉质量评分上出现稳定退化,且推理计算开销显著增加。

针对上述问题,本项目拟研究多智能体协同的空间感知图像恢复方法,实现精细化、高效化的复杂退化图像恢复,形成“可感知、可决策、可迭代、可部署”的新型恢复范式。

从研究意义看,本项目可归纳为以下三个层面:

一是\textbf{学术意义}:推动图像恢复从“全图单策略处理”向“区域级动态决策处理”演进,丰富复杂退化图像恢复的理论内涵。

二是\textbf{方法意义}:促进空间感知、智能体规划和轨迹蒸馏的深度融合,形成兼顾恢复质量与计算效率的方法体系。

三是\textbf{应用与战略意义}:面向安防巡检、智能交通、遥感解译等场景提供高可靠视觉基础能力,服务智能感知系统的工程落地。

\subsection{国内外研究现状}

根据方法论的演进,现有图像恢复方法可分为三类。

为直观对比“传统全图统一处理”与“本项目空间感知分区处理”的差异,图0给出了方法框架层面的示意对照。

\vspace{4pt}
\begin{center}
\includegraphics[width=0.92\linewidth]{imgs/Gemini_Generated_Image_lk4hu6lk4hu6lk4h.png}\\[2pt]
{\small 图0\quad 传统全图统一恢复与空间感知分区恢复的框架对照示意}
\end{center}
\vspace{4pt}

(1)基于单任务模型的图像恢复算法

这类方法针对特定退化类型设计专用网络。Liang等人\cite{swinir}提出的SwinIR基于Swin Transformer实现了去噪、超分辨率等任务;Zamir等人\cite{restormer}提出的Restormer通过高效Transformer架构在去噪、去模糊等任务上取得了优异性能。这类方法在各自任务上表现出色,但需为每种退化单独训练模型,且无法处理多退化共存的复杂场景。

(2)全能型图像恢复算法

为克服单任务模型的局限性,研究者尝试构建统一模型处理多种退化。Li等人\cite{airnet}提出的AirNet基于对比学习实现退化表征的统一处理;Potlapalli等人\cite{promptir}提出的PromptIR通过可学习的提示向量隐式编码退化信息;Conde等人\cite{instructir}提出的InstructIR首次利用自然语言指令引导图像恢复。然而,这些方法受限于训练阶段所覆盖的退化类型(通常3--7种),面对复杂混合退化时性能显著下降,且采用单次前向传播的静态处理模式,无法迭代优化恢复结果。

(3)基于智能体的图像恢复算法

基于智能体的方法利用大语言模型(LLM)与视觉语言模型(VLM)智能编排专用恢复工具,代表着图像恢复领域的新范式。Zhu等人\cite{agenticir}提出的AgenticIR是该领域的开创性工作,设计了感知---调度---执行---反思---重调度的五阶段框架,但采用深度优先搜索,计算开销极大。Chen等人\cite{restoreagent}提出的RestoreAgent通过微调多模态大模型直接预测恢复流程。Jiang等人\cite{mair}提出的MAIR引入多智能体架构和三阶段退化先验,将推理效率提升44\%。Zhou等人\cite{qagent}提出的Q-Agent采用质量驱动的贪心策略,实现线性复杂度的恢复规划。

尽管上述方法取得了重要进展,但均存在以下共性不足:所有方法对整张图像进行全局统一处理,忽略了退化在空间上的不均匀分布;工具调用仅限于二元选择,缺乏细粒度的参数调控;推理阶段对大模型的强依赖导致计算代价居高不下。这些问题构成了本项目的核心研究动机。

\textbf{当前研究存在的主要问题可归纳为:}

一、\textbf{退化表征层面}:现有方法对空间异质退化建模不足,难以在区域粒度上准确识别“退化类型—严重程度—位置分布”的耦合关系。

二、\textbf{恢复决策层面}:多数方法缺乏“工具链+参数”的联合优化机制,难以实现面向复杂场景的动态规划与迭代修复。

三、\textbf{部署效率层面}:对LLM/VLM等大模型推理依赖较强,导致时延和算力成本偏高,不利于边缘部署与实时应用。

\subsection{本项目的应用价值}

本项目围绕“复杂退化图像高质量恢复”这一核心需求,面向典型视觉任务链条中的前置质量增强环节,具有明确应用价值。

一是\textbf{行业场景价值}:在智慧安防与交通场景中,本项目可提升夜间/恶劣天气图像可用性,增强后续检测、跟踪、识别模块的稳定性;在遥感场景中,可提升地物边界与纹理细节质量,服务变化检测与精细解译任务。

二是\textbf{技术迁移价值}:本项目提出的“空间退化图 + 层级智能体 + 轨迹蒸馏”框架具备较强任务迁移潜力,可推广至视频增强、医学影像恢复、工业视觉质检等相关方向。

三是\textbf{组织与资源匹配价值}:依托既有图像恢复研究积累、可复用模型工具链与实验平台,本项目在数据、算法与工程实现层面具备较好的实施基础,能够实现从研究原型到可验证系统的落地转化。

本节小结:通过上述选题依据与价值分析,可进一步明确本项目需在“目标可量化、框架可执行、难点可攻关”的前提下展开研究,因此第2节将围绕研究目标、总体框架与重点难点进行系统展开。


\section{研究目标、总体框架及重点难点}

本节围绕“目标—内容—难点”建立一一映射关系:目标定义“做成什么”,研究内容定义“怎么做”,重点难点定义“最难在哪里以及如何突破”,三者共同支撑项目可执行性。

\subsection{研究目标}

面向复杂退化场景下图像恢复的精细化与高效化需求,本项目拟实现以下可评价目标:

1)\textbf{目标1(感知)}:构建空间感知退化分析机制,实现区域级退化识别与量化;在公开复杂退化基准上,相较全图级识别基线,区域级退化识别准确率显著提升。对应输出物:空间退化图构建算法、区域级标注与评测脚本、阶段论文成果。

2)\textbf{目标2(决策)}:构建层级多智能体协同机制,实现区域级恢复规划与连续参数自适应优化;在混合退化恢复任务中,相较静态恢复管线获得更高恢复质量与更优效率平衡。对应输出物:层级智能体恢复系统原型、工具链调度策略、发明专利/论文成果。

3)\textbf{目标3(部署)}:构建轨迹蒸馏驱动的轻量路由模型,实现高质量恢复策略的快速推理与边缘部署;在保持主要质量指标可接受损失范围内显著降低部署时延与算力需求。对应输出物:轻量路由网络模型、部署测试报告、可复现实验配置。

\subsection{总体框架}

本项目总体框架如图1所示,由”空间感知模块—层级规划模块—轻量部署模块”三大核心模块组成。空间感知模块负责将退化图像分割为语义区域并生成空间退化图;层级规划模块基于全局调度器与区域专家的协同架构,对各区域独立规划恢复策略并迭代优化;轻量部署模块通过轨迹蒸馏将多智能体的复杂决策知识压缩为轻量路由网络。三个模块层层递进,形成”感知—决策—执行—蒸馏”的闭环流程。

其中,系统输入为退化图像$\mathbf{y}$;模块一输出空间退化图$\mathcal{M}$;模块二输出区域修复结果及全图恢复结果$\hat{\mathbf{x}}$;模块三输出轻量推理模型$\Phi_\omega$及对应调度参数。三模块在训练阶段通过轨迹数据闭环耦合,在部署阶段形成“轻量网络 + 工具库”的高效执行路径。

% === 图1 占位符 ===
% 【画图 AI Prompt(详细描述)】:
% 绘制一张学术论文风格的水平流程图,展示”多智能体协同的空间感知图像恢复系统”的总体框架。白色背景,专业扁平化设计,无3D效果。
%
% 整体布局:从左到右的水平管线流程,分为输入、三个核心模块、输出五个部分。
%
% 【输入端(最左侧)】:一张退化图像,图像中不同空间区域呈现不同的退化类型——
%   左上区域有明显的高斯噪声颗粒,中部区域有运动模糊拖影,下方区域被雾霾笼罩呈灰白色,
%   右侧有雨线条纹。图像下方标注”退化图像 y”。
%
% 【模块一(浅蓝色圆角矩形,标题栏写”空间感知模块”)】:
%   内部自上而下排列三个步骤,用箭头串联:
%   (a) “SAM分割”:图标为一个分割网络示意,旁边显示退化图像被分割为4-5个彩色区域
%       (天空=浅蓝、建筑=橙色、道路=灰色、植被=绿色),区域边界用彩色虚线标出。
%   (b) “VLM退化分析”:图标为DepictQA模型,旁边用气泡框展示链式推理过程,
%       例如”区域1:噪声=强,模糊=无,雾霾=中等”。
%   (c) 输出:一张”空间退化图 M”,用热力图/彩色编码叠加在原图上,
%       不同颜色代表不同退化类型(红色=噪声,蓝色=模糊,灰色=雾霾,黄色=雨线),
%       颜色深浅表示严重程度。旁边标注数学符号 M = {(R_i, d_i, s_i)}。
%
% 【模块二(浅橙色圆角矩形,标题栏写”层级规划模块”)】:
%   内部为两层架构:
%   (上层) “全局调度器”:一个大的控制节点,图标为大脑/指挥中心,标注”LLM”。
%       旁边显示退化先验排序规则:”压缩退化→成像退化→场景退化”(逆序处理箭头)。
%       从空间退化图 M 有箭头输入到全局调度器。
%   (下层) 3-4个”区域专家”:每个为一个小智能体图标,分别标注”Expert 1””Expert 2””Expert 3”。
%       全局调度器与各区域专家之间有双向箭头(指令下发↓,结果上报↑)。
%       每个区域专家连接一个”工具库”(竖排图标列表:去噪器Denoiser、去模糊器Deblurrer、
%       去雾器Dehazer、去雨器Derainer、超分辨率SR),每个工具旁有参数滑块(θ)。
%   (反馈环) 从修复结果有箭头指向”质量评估 Q(·)”模块,再反馈回区域专家,
%       形成闭环,标注”迭代优化”。
%   (输出) 各区域修复结果通过”掩码融合 ⊕”节点合成为完整修复图像。
%
% 【模块三(浅绿色圆角矩形,标题栏写”轻量部署模块”)】:
%   (上方) 从模块二有一条虚线箭头标注”离线轨迹收集”,指向多组”(输入图像→决策轨迹)”
%       数据对的示意图标。
%   (中部) “轨迹蒸馏”箭头指向一个轻量级CNN网络结构示意图,标注”路由网络 Φ_ω”。
%   (下方) 一个被红叉划掉的LLM图标,旁边文字”无需大模型”,表示部署时不依赖LLM/VLM。
%       路由网络直接输出:区域掩码 + 退化类型 + 工具路由 + 参数。
%
% 【输出端(最右侧)】:一张清晰的高质量恢复图像,各区域退化均已消除,
%   色彩鲜明,细节清晰。图像下方标注”恢复图像 x̂”。
%
% 【全局连接】:三个模块之间用粗实线箭头顺序连接。模块一→模块二标注”空间退化图”,
%   模块二→输出标注”恢复图像”。模块二到模块三用虚线箭头标注”轨迹数据”。
%   模块三→输出有一条虚线标注”实时推理路径(部署阶段)”。
%
\vspace{6pt}
\begin{center}
\includegraphics[width=\linewidth]{imgs/Gemini_Generated_Image_lk4hu6lk4hu6lk4h.png}\\[2pt]
{\small 图1\quad 本项目总体研究框架}
\end{center}
\vspace{6pt}

\subsection{研究内容}

面向上述目标,本项目拟开展以下三个方面的研究。

\textbf{1)空间感知退化分析}

研究问题:现有方法主要基于全图级退化判断,难以刻画复杂场景的空间异质退化。关键方法:引入零样本分割模型SAM\cite{sam}进行语义区域划分,并结合DepictQA\cite{depictqa}进行区域级退化类型与严重程度估计,构建空间退化图。预期效果:为后续规划提供可解释、可量化的结构化输入。与研究内容2的依赖关系:本模块输出$\mathcal{M}$作为层级智能体的决策先验。创新嵌入点:以区域粒度统一表征“退化类型-程度-位置”。

\textbf{2)层级多智能体协同规划}

研究问题:单一静态恢复流程难以兼顾复杂退化下的质量与效率。关键方法:构建“全局调度器 + 区域专家”的两层智能体架构,全局层依据退化先验\cite{mair}分配处理优先级,区域层执行工具链选择与连续参数优化,并通过质量反馈迭代更新。预期效果:实现区域差异化修复与全图一致性兼顾。与研究内容1/3的依赖关系:依赖$\mathcal{M}$完成规划,并为轨迹蒸馏提供高质量决策样本。创新嵌入点:实现工具调用与参数调节的一体化决策。

\textbf{3)基于轨迹蒸馏的轻量路由网络}

研究问题:多智能体高质量推理成本高、部署门槛高。关键方法:离线收集“输入图像$\to$最优路径与参数”轨迹数据,训练轻量路由网络直接预测分割、退化类型与工具调度方案。预期效果:在保持恢复质量可接受范围内大幅降低部署时延与算力开销。与研究内容2的依赖关系:蒸馏对象来自多智能体轨迹;反向促进规划策略标准化。创新嵌入点:将复杂多步推理知识压缩为可部署决策网络。

\subsection{重点难点}

本项目的重点难点包括:

1)\textbf{难点1(对应目标1)}:如何在零样本条件下准确分割退化区域并量化退化程度。拟解决路径:通过“语义分割 + 区域级退化推理 + 一致性约束”联合优化空间退化图质量。

2)\textbf{难点2(对应目标2)}:如何在空间维度上协调多智能体决策,避免区域边界不一致与策略冲突。拟解决路径:通过全局调度优先级约束与区域专家反馈闭环实现跨区域协同。

3)\textbf{难点3(对应目标3)}:如何将复杂多步推理知识有效压缩到轻量网络中并保持性能稳定。拟解决路径:采用多任务联合蒸馏、轨迹筛选与部署约束联合训练。

以上难点分别对应“感知—决策—部署”三类核心目标,并在第3节中给出具体技术实现路径。


\section{思路方法}

\subsection{具体研究方法}

本项目的整体研究思路为“感知$\to$规划$\to$执行$\to$蒸馏”。首先构建空间感知模块实现区域级退化分析,然后设计层级智能体进行协同规划与执行,最后通过轨迹蒸馏实现轻量化部署。三个研究内容层层递进、相互依托。具体方法流程如图~\ref{fig:method-pipeline}~所示。

% === 图2 占位符 ===
% 【画图 AI Prompt(详细描述)】:
% 绘制一张学术论文风格的详细方法流程图,展示"多智能体协同的空间感知图像恢复"的
% 完整技术流程。白色背景,专业扁平化设计,整体布局为从上到下的三行结构,
% 行与行之间有清晰的数据流箭头连接。
%
% 【第一行:空间感知退化分析(浅蓝色背景带)】
%   从左到右排列:
%   (a) 输入:一张混合退化图像 y(城市街景,不同区域有不同退化:
%       天空区域有雾霾、建筑区域有噪声、道路区域有模糊、植被区域有雨线)。
%   (b) SAM分割步骤:图像被SAM模型分割为4-5个语义区域,
%       每个区域用不同颜色的半透明掩码覆盖(天空=蓝色掩码、建筑=橙色掩码、
%       道路=灰色掩码、植被=绿色掩码),区域边界用白色线条标出。
%       上方标注公式 "{R_i} = SAM(y)"。
%   (c) VLM逐区域分析步骤:展示为多个并行的分析通道,每个通道对应一个区域。
%       每个通道显示:区域裁剪图 → DepictQA模型图标 → 结构化输出。
%       用文字气泡展示分析结果示例:
%       "R1(天空): 雾霾=0.8, 噪声=0.1, 模糊=0.0"
%       "R2(建筑): 噪声=0.9, 雾霾=0.2, 模糊=0.1"
%       "R3(道路): 模糊=0.7, 噪声=0.3, 雾霾=0.0"
%       上方标注 "(d_i, s_i) = VLM(y_{R_i})"。
%   (d) 输出:空间退化图 M,以彩色热力图叠加在原图上展示,
%       每个区域标注其主要退化类型和严重程度数值。
%       旁边用数学符号标注 "M = {(R_i, d_i, s_i)}"。
%
% 【第二行:层级多智能体协同规划(浅橙色背景带)】
%   中央为双层架构示意:
%   (上层——全局调度器) 一个大的圆角矩形节点,内部显示:
%       左侧:接收空间退化图 M 作为输入(从第一行有粗箭头连下来)。
%       中部:退化先验排序规则的可视化——三个箭头从右到左排列,
%       分别标注"1.压缩退化(JPEG伪影)""2.成像退化(噪声/模糊)""3.场景退化(雾霾/雨线)",
%       表示逆序处理优先级。
%       右侧:输出各区域的处理优先级队列和策略分配。
%   (下层——区域专家组) 3个并排的小圆角矩形,分别标注"区域专家1""区域专家2""区域专家3"。
%       每个专家内部显示:
%       - 工具选择:从一个竖排的"工具库"(包含图标:Denoiser去噪、Deblurrer去模糊、
%         Dehazer去雾、Derainer去雨、SR超分辨率、JPEG去伪影)中选出2-3个工具,
%         用箭头连成工具链,例如 Expert1: Dehazer→Denoiser。
%       - 参数设定:每个选中工具旁有一个小滑块/旋钮,标注连续参数值
%         如 "σ=0.7""scale=2×"。
%       - 工具链右侧显示修复后的区域图像小缩略图。
%   (反馈回路) 每个区域专家的输出有箭头指向一个"质量评估 Q(·)"模块
%       (DepictQA图标),评估模块输出分数(如"Q=0.82"),
%       与阈值 τ 比较:若 Q < τ 则红色箭头反馈回区域专家(标注"调参重试"),
%       若 Q ≥ τ 则绿色箭头向右输出(标注"通过")。
%   (融合输出) 所有通过质量阈值的区域修复结果汇聚到一个"掩码加权融合 ⊕"节点,
%       输出最终恢复图像 x̂。公式标注 "x̂ = Σ R_i ⊙ x̂_{R_i}"。
%
% 【第三行:轨迹蒸馏与轻量部署(浅绿色背景带)】
%   左侧——数据收集:
%       从第二行有一条大的虚线箭头标注"离线运行 & 轨迹记录",
%       指向多组轨迹数据的示意:3-4个小卡片,每个卡片左侧是退化图像缩略图,
%       右侧是该图像对应的决策轨迹(区域划分 + 工具链 + 参数),
%       用文字简写如 "y→{R1:Dehaze(0.8)→Denoise(0.5), R2:Deblur(0.6)→SR(2×)}"。
%       标注 "轨迹数据集 T, |T|=M"。
%   中部——蒸馏训练:
%       一个"知识蒸馏"箭头(粗虚线箭头,标注"监督学习")从轨迹数据指向
%       一个轻量级网络结构示意图。该网络以退化图像为输入(左侧),
%       经过一个小型编码器(几层卷积块),输出四个分支头:
%       "分割头"(输出区域掩码)、"分类头"(输出退化类型)、
%       "路由头"(输出工具选择)、"参数头"(输出连续参数)。
%       网络标注为 "Φ_ω"。损失函数标注在网络下方:
%       "L = L_seg + λ1·L_cls + λ2·L_route + λ3·L_param"。
%   右侧——部署推理:
%       轻量路由网络 Φ_ω 直接连接工具库执行恢复,输出最终图像。
%       旁边标注关键优势:"无需LLM/VLM"(旁有一个被划掉的大模型图标)、
%       "实时推理"(旁有一个时钟/闪电图标)。
%
% 【全局标注】:
%   三行左侧分别标注研究内容编号 "(1)""(2)""(3)"。
%   行与行之间的连接箭头标注传递的数据名称。
%   整体配色:蓝色系(感知)、橙色系(规划)、绿色系(部署),
%   与图1的三模块颜色保持一致。
%
\vspace{6pt}
\begin{figure}[H]
\centering
\includegraphics[width=\linewidth]{imgs/Gemini_Generated_Image_qjhpc4qjhpc4qjhp.png}\\[2pt]
\caption{具体研究方法流程}
\label{fig:method-pipeline}
\end{figure}
\vspace{6pt}

\textbf{(1)空间感知退化分析}

给定退化图像$\mathbf{y} \in \mathbb{R}^{H \times W \times 3}$,本模块旨在生成空间退化图$\mathcal{M}$,为后续智能体规划提供结构化输入。

\textbf{步骤一:语义区域分割。}利用零样本分割模型SAM\cite{sam}将图像划分为$N$个语义一致的区域:
\begin{equation}
\{R_i\}_{i=1}^{N} = \mathrm{SAM}(\mathbf{y}), \quad \bigcup_{i=1}^{N} R_i = \Omega, \quad R_i \cap R_j = \emptyset \ (i \neq j)
\end{equation}
其中$\Omega$为图像像素域。SAM基于超过10亿掩码的预训练,具备良好的零样本泛化能力。然而,直接应用于退化图像时,模糊边界、噪声纹理和对比度损失会导致分割质量下降。为此,本研究引入置信度门控机制:
\begin{equation}
R_i^{\text{final}} = \begin{cases} R_i, & \text{if } \mathcal{C}(R_i) > \tau_{\text{seg}} \\ R_i^{\text{robust}}, & \text{otherwise} \end{cases}
\end{equation}
其中$\mathcal{C}(R_i)$为SAM对区域$R_i$的分割置信度,$\tau_{\text{seg}}$为预设阈值。当置信度低于阈值时,启用RobustSAM\cite{robustsam}重新分割,其在噪声、模糊、低对比度等退化条件下保持稳健性能。

\textbf{步骤二:区域级退化分析。}对每个分割区域$R_i$,提取其裁剪图像$\mathbf{y}_{R_i} = \mathbf{y} \odot \mathbf{M}_i$,其中$\mathbf{M}_i \in \{0,1\}^{H \times W}$为区域掩码。将$\mathbf{y}_{R_i}$输入经Q-Instruct数据集微调的视觉语言模型$V_{\theta}$,采用链式思维推理逐类判断退化:
\begin{equation}
\mathbf{d}_i^{(c)}, s_i^{(c)} = V_{\theta}\left(\text{Prompt}_c, \mathbf{y}_{R_i}\right), \quad c = 1, \ldots, C
\end{equation}
其中$\text{Prompt}_c$为针对第$c$类退化的查询提示。为提升分析精度,引入空间上下文聚合:
\begin{equation}
\mathbf{h}_i = \text{ContextAgg}\left(\{V_{\theta}(\mathbf{y}_{R_j})\}_{j \in \mathcal{N}(i)}\right)
\end{equation}
其中$\mathcal{N}(i)$表示区域$R_i$的空间邻域集合,通过图注意力机制聚合邻域退化信息,修正孤立区域的误判。

\textbf{步骤三:空间退化图构建。}汇总所有区域分析结果,构建结构化空间退化图:
\begin{equation}
\mathcal{M} = \left\{(R_i, \mathbf{d}_i, \mathbf{s}_i, \mathbf{h}_i)\right\}_{i=1}^{N}
\end{equation}
该图显式编码了退化的空间分布、类型判别与严重程度估计,可直接作为后续决策模块的状态描述。

\textbf{验证方案:}在SIDD去噪、GoPro去模糊、NH-HAZE去雾等公开基准上,采用分类准确率(Acc)、平均精度(mAP)和F1分数评估退化识别性能。对比方案包括:(1)全图级DepictQA分析基线;(2)SAM分割 + 独立区域分析(无上下文聚合);(3)本研究完整方法。通过消融实验量化各组件(RobustSAM回退、上下文聚合)对最终恢复质量(PSNR/SSIM)的贡献。

\textbf{(2)层级多智能体协同规划}

基于空间退化图$\mathcal{M}$,本研究设计全局调度器$G_{\phi}$与区域专家$\{E_i\}_{i=1}^{N}$的协同架构。全局调度器负责跨区域策略协调,区域专家负责区域内恢复决策。

\textbf{步骤一:全局调度与优先级分配。}全局调度器依据退化先验(场景退化$\to$成像退化$\to$压缩退化的逆序处理原则\cite{mair}),为各区域分配处理优先级。定义退化优先级函数:
\begin{equation}
\mathcal{P}(R_i) = \sum_{c=1}^{C} w_c \cdot d_i^{(c)} \cdot s_i^{(c)}
\end{equation}
其中$w_c$为第$c$类退化的先验权重(场景退化$w_c > 1$,成像退化$w_c = 1$,压缩退化$0 < w_c < 1$)。全局调度器按$\mathcal{P}(R_i)$降序排列区域处理顺序,并生成全局策略向量:
\begin{equation}
\mathbf{g} = G_{\phi}(\mathcal{M}) = \text{Softmax}\left(\text{MLP}\left(\{\mathbf{h}_i\}_{i=1}^{N}\right)\right)
\end{equation}
其中$\mathbf{g} \in \mathbb{R}^{N}$编码各区域的处理优先级权重。

\textbf{步骤二:区域专家工具链选择与参数优化。}对于区域$R_i$,区域专家$E_i$从预构建的工具库$\mathcal{F} = \{f_1, f_2, \ldots, f_L\}$中选择恢复工具序列$\pi_i = (f_{\sigma_1}, \ldots, f_{\sigma_{K_i}})$,并为每个工具设定连续参数$\boldsymbol{\theta}_i = (\theta_{\sigma_1}, \ldots, \theta_{\sigma_{K_i}})$。区域修复结果通过函数复合计算:
\begin{equation}
\hat{\mathbf{x}}_{R_i} = (f_{\sigma_{K_i}} \circ \cdots \circ f_{\sigma_1})(\mathbf{y}_{R_i}; \boldsymbol{\theta}_i)
\end{equation}

工具链选择采用策略梯度方法优化。定义策略网络$\pi_{\psi}(\cdot | \mathbf{h}_i, \mathbf{g})$输出工具选择概率分布,通过最大化期望恢复质量进行训练:
\begin{equation}
\max_{\psi} \mathbb{E}_{\pi_i \sim \pi_{\psi}}\left[Q\left((f_{\sigma_{K_i}} \circ \cdots \circ f_{\sigma_1})(\mathbf{y}_{R_i})\right)\right] - \beta \cdot \text{Length}(\pi_i)
\end{equation}
其中$Q(\cdot)$为质量评估函数,$\beta$为工具链长度惩罚系数,鼓励简洁高效的处理流程。

\textbf{步骤三:连续参数自适应优化。}对于选定的工具链$\pi_i$,区域专家通过迭代优化确定最优参数:
\begin{equation}
\boldsymbol{\theta}_i^{*} = \arg\max_{\boldsymbol{\theta}_i} Q\left(\hat{\mathbf{x}}_{R_i}(\boldsymbol{\theta}_i)\right), \quad \text{s.t.} \quad Q\left(\hat{\mathbf{x}}_{R_i}(\boldsymbol{\theta}_i)\right) \geq \tau
\end{equation}
其中$\tau$为质量阈值。采用贝叶斯优化或梯度上升法求解,迭代更新直至收敛或达到最大迭代次数$T_{\max}$。

\textbf{步骤四:全局融合与一致性约束。}全局调度器通过区域掩码加权融合各区域结果:
\begin{equation}
\hat{\mathbf{x}} = \sum_{i=1}^{N} \mathbf{M}_i \odot \hat{\mathbf{x}}_{R_i} + \lambda_{\text{smooth}} \cdot \nabla^2 \hat{\mathbf{x}}
\end{equation}
其中$\lambda_{\text{smooth}}$为平滑正则系数,通过拉普拉斯算子约束边界一致性。全局质量评估通过一致性损失实现跨区域协调:
\begin{equation}
\mathcal{L}_{\text{consist}} = \sum_{(i,j) \in \mathcal{E}} \left\| \hat{\mathbf{x}}_{R_i}^{\text{border}} - \hat{\mathbf{x}}_{R_j}^{\text{border}} \right\|_2^2
\end{equation}
其中$\mathcal{E}$为相邻区域边界的像素集合。

\textbf{验证方案:}在混合退化合成数据集(通过SIDD、GoPro、NH-HAZE等组合生成)与真实测试样本上,与AirNet、PromptIR等All-in-One静态方法以及AgenticIR、MAIR等智能体方法进行全面对比。评估指标包括:(1)恢复质量:PSNR、SSIM、LPIPS;(2)推理效率:FPS、平均工具调用次数;(3)决策质量:工具链选择准确率、参数优化收敛速度。通过消融实验验证层级架构各组件(全局调度、区域专家、质量反馈、一致性约束)的贡献度。

\textbf{(3)基于轨迹蒸馏的轻量路由网络}

\textbf{步骤一:决策轨迹数据集构建。}利用上述多智能体系统在大规模合成退化数据集$\mathcal{D}_{\text{syn}} = \{(\mathbf{y}^{(j)}, \mathbf{x}^{(j)})\}_{j=1}^{M}$上离线运行,收集决策轨迹数据集:
\begin{equation}
\mathcal{T} = \left\{\left(\mathbf{y}^{(j)}, \mathcal{M}^{(j)}, \left\{\pi_i^{*,(j)}, \boldsymbol{\theta}_i^{*,(j)}, \mathbf{M}_i^{(j)}\right\}_{i=1}^{N_j}\right)\right\}_{j=1}^{M}
\end{equation}
其中每条轨迹记录输入图像$\mathbf{y}^{(j)}$、空间退化图$\mathcal{M}^{(j)}$、各区域最优工具序列$\pi_i^{*}$、最优参数$\boldsymbol{\theta}_i^{*}$以及区域掩码$\mathbf{M}_i$。

为提升轨迹质量,引入结果过滤机制:仅保留最终恢复质量$Q(\hat{\mathbf{x}}^{(j)}) \geq \tau_{\text{high}}$的轨迹参与蒸馏。同时,采用轨迹多样性采样确保退化类型、区域数量、工具链长度的分布均衡。

\textbf{步骤二:轻量路由网络架构设计。}设计轻量级路由网络$\Phi_{\omega}$,采用编码器—多任务解码器架构:

编码器采用轻量化骨干网络(如MobileNet-V3或EfficientNet-B0)提取多尺度特征金字塔$\{\mathbf{F}_l\}_{l=1}^{L}$,其中$\mathbf{F}_l \in \mathbb{R}^{H/2^l \times W/2^l \times C_l}$。

解码器分为四个并行分支头:

(a) \textbf{分割头}$\Phi_{\omega}^{\text{seg}}$:输出区域掩码概率图$\hat{\mathbf{P}}^{\text{mask}} \in \mathbb{R}^{H \times W \times N_{\max}}$,通过Softmax获取区域归属概率:
\begin{equation}
\hat{\mathbf{M}}_i = \mathbb{1}\left[\arg\max_k \hat{P}^{\text{mask}}(u,v,k) = i\right]
\end{equation}

(b) \textbf{分类头}$\Phi_{\omega}^{\text{cls}}$:对每个区域输出退化类型概率分布:
\begin{equation}
\hat{\mathbf{d}}_i = \text{Sigmoid}\left(\text{GAP}\left(\mathbf{F}_{\text{roi}}^{(i)}\right)\right) \in [0,1]^{C}
\end{equation}
其中$\text{GAP}$为全局平均池化,$\mathbf{F}_{\text{roi}}^{(i)}$为区域$R_i$对应的RoI特征。

(c) \textbf{路由头}$\Phi_{\omega}^{\text{route}}$:输出工具选择概率矩阵:
\begin{equation}
\hat{\mathbf{P}}^{\text{route}}_i = \text{Softmax}\left(\text{MLP}\left(\mathbf{F}_{\text{roi}}^{(i)} \oplus \hat{\mathbf{d}}_i\right)\right) \in \mathbb{R}^{L}
\end{equation}
其中$\oplus$表示特征拼接。

(d) \textbf{参数头}$\Phi_{\omega}^{\text{param}}$:对每个选中工具输出连续参数值:
\begin{equation}
\hat{\boldsymbol{\theta}}_i = \text{Sigmoid}\left(\text{MLP}\left(\mathbf{F}_{\text{roi}}^{(i)}\right)\right) \cdot \boldsymbol{\theta}_{\max}
\end{equation}
其中$\boldsymbol{\theta}_{\max}$为各参数的最大允许值。

\textbf{步骤三:多任务联合蒸馏训练。}基于轨迹数据集进行监督学习,联合优化四个任务:
\begin{equation}
\mathcal{L} = \mathcal{L}_{\text{seg}} + \lambda_1 \mathcal{L}_{\text{cls}} + \lambda_2 \mathcal{L}_{\text{route}} + \lambda_3 \mathcal{L}_{\text{param}} + \lambda_4 \mathcal{L}_{\text{kd}}
\end{equation}

各损失项定义如下:
\begin{align}
\mathcal{L}_{\text{seg}} &= \frac{1}{M} \sum_{j=1}^{M} \sum_{i=1}^{N_j} \text{Dice}\left(\hat{\mathbf{M}}_i^{(j)}, \mathbf{M}_i^{*,(j)}\right) \\
\mathcal{L}_{\text{cls}} &= \frac{1}{M} \sum_{j=1}^{M} \sum_{i=1}^{N_j} \text{BCE}\left(\hat{\mathbf{d}}_i^{(j)}, \mathbf{d}_i^{*,(j)}\right) \\
\mathcal{L}_{\text{route}} &= \frac{1}{M} \sum_{j=1}^{M} \sum_{i=1}^{N_j} \sum_{k=1}^{K_i} \text{CE}\left(\hat{\mathbf{P}}^{\text{route}}_{i,k}, f_{\sigma_k}^{*,(j)}\right) \\
\mathcal{L}_{\text{param}} &= \frac{1}{M} \sum_{j=1}^{M} \sum_{i=1}^{N_j} \sum_{k=1}^{K_i} \left\|\hat{\boldsymbol{\theta}}_{i,k}^{(j)} - \boldsymbol{\theta}_{i,k}^{*,(j)}\right\|_2^2
\end{align}
其中$\lambda_1, \lambda_2, \lambda_3, \lambda_4$为任务平衡系数。知识蒸馏损失$\mathcal{L}_{\text{kd}}$用于保留多智能体的“软”决策信息:
\begin{equation}
\mathcal{L}_{\text{kd}} = \text{KL}\left(\text{Softmax}(\hat{\mathbf{z}} / T) \| \text{Softmax}(\mathbf{z}^{*} / T)\right)
\end{equation}
其中$\hat{\mathbf{z}}$和$\mathbf{z}^{*}$分别为路由网络和多智能体的logits输出,$T$为温度系数。

\textbf{步骤四:困难样本再训练与部署优化。}针对蒸馏偏差问题,实施两阶段训练策略:

第一阶段:在全量轨迹数据上训练至收敛。

第二阶段:识别困难样本集$\mathcal{T}_{\text{hard}} = \{(\mathbf{y}^{(j)}, \cdot) \in \mathcal{T} : Q(\hat{\mathbf{x}}^{(j)}) < Q(\mathbf{x}_{\text{agent}}^{(j)}) - \delta\}$,即路由网络与多智能体性能差距超过阈值$\delta$的样本。在该子集上进行困难样本聚焦训练,损失函数加权:
\begin{equation}
\mathcal{L}_{\text{hard}} = \sum_{(\mathbf{y}^{(j)}, \cdot) \in \mathcal{T}_{\text{hard}}} w_j \cdot \mathcal{L}^{(j)}, \quad w_j \propto Q(\mathbf{x}_{\text{agent}}^{(j)}) - Q(\hat{\mathbf{x}}^{(j)})
\end{equation}

部署阶段,轻量路由网络直接根据输入图像预测区域划分、退化类型、工具调度与参数配置,无需LLM/VLM在线推理,实现端到端实时恢复:
\begin{equation}
\left\{\hat{\mathbf{M}}_i, \hat{\mathbf{d}}_i, \hat{\pi}_i, \hat{\boldsymbol{\theta}}_i\right\}_{i=1}^{\hat{N}} = \Phi_{\omega}(\mathbf{y})
\end{equation}

\textbf{验证方案:}在测试集上对比蒸馏后的轻量路由网络与原多智能体系统的性能差距,要求PSNR损失$\leq 5\%$、SSIM损失$\leq 3\%$。测试部署时延(ms/图像)、模型参数量(MB)、计算量(FLOPs),验证在Jetson Nano或同等边缘设备上的实时推理可行性(目标:$\geq 10$ FPS @ 720p)。通过消融实验验证困难样本再训练对性能提升的贡献,分析不同退化类型、区域数量对蒸馏效果的影响。


\subsection{研究计划}

本项目的计划研究时间为2年,具体研究计划如下:

\textbf{第一阶段(第1–6月)}

\begin{itemize}
\item 完成空间感知退化分析的基础架构,实现SAM分割与VLM退化识别的联调;
\item 引入上下文聚合与RobustSAM回退机制,建立区域级退化识别评测体系;
\item 在SIDD、GoPro、NH-HAZE等基准上验证退化类型识别准确率;
\end{itemize}

\textbf{第二阶段(第7–12月)}

\begin{itemize}
\item 完成层级多智能体协同架构搭建,实现全局调度器与区域专家的基础功能;
\item 引入全局一致性约束与早停机制,与AirNet、PromptIR等方法完成系统性对比;
\item 构建混合退化合成数据集,完成发明专利交底书撰写;
\item 向本领域国内外权威学术期刊或重要学术会议投稿论文1–2篇;
\end{itemize}

\textbf{第三阶段(第13–18月)}

\begin{itemize}
\item 构建规模不少于10000条的多智能体决策轨迹数据集;
\item 完成轻量路由网络的多任务联合蒸馏,实现端到端推理基础版本;
\item 针对困难样本进行聚焦训练,缩小与多智能体系统的性能差距;
\end{itemize}

\textbf{第四阶段(第19–24月)}

\begin{itemize}
\item 完成轻量路由网络的部署优化,在边缘设备(Jetson Nano或同等级平台)完成测试;
\item 实现系统集成,形成完整的感知—规划—执行—蒸馏闭环;
\item 向本领域国内外权威学术期刊或重要学术会议投稿论文1–2篇;
\item 撰写项目结题报告,汇总全部研究成果。
\end{itemize}

\subsection{可行性分析}

\textbf{(1)理论可行性}

多智能体协同与大模型规划机制已在复杂任务求解中证明有效\cite{agenticir,restoreagent,mair,qagent};空间感知分析所依赖的SAM\cite{sam}与DepictQA\cite{depictqa}具备成熟的预训练基础;轨迹蒸馏作为知识压缩手段已在多任务学习中展现稳定性。上述理论基础与本项目“感知—规划—蒸馏”技术路线一致。

\textbf{(2)技术可行性}

在团队基础方面,已形成图像恢复与连续比例因子建模研究积累\cite{wu2023tgrs,wu2022cjig,wu2020tgrs};在平台条件方面,具备预训练模型接入、训练评测与可视化分析环境;在数据条件方面,具备构建合成退化与真实测试样本的能力。针对潜在瓶颈(算力开销、数据噪声、模型稳定性),已预设轻量路由替代、数据清洗与困难样本重训等缓释策略。

综上,本项目在理论与工程两个层面均具备实施条件,可支撑既定目标按计划推进。

\section{创新之处}

本项目创新点按“理论创新—方法创新—技术创新”组织如下:

一是\textbf{理论创新}:提出“空间退化状态驱动的恢复决策”框架,将图像恢复问题从全图单步映射扩展为区域状态感知下的多步决策问题。相对已有方法的增量边界在于:由“统一恢复函数”转向“状态条件策略函数”。

二是\textbf{方法创新}:提出全局调度器与区域专家协同的层级多智能体机制,实现工具链选择与连续参数优化一体化。相对已有方法的增量边界在于:从离散工具选择扩展到“离散路由 + 连续调参”的联合优化。

三是\textbf{技术创新}:提出基于轨迹蒸馏的轻量路由部署方案,将高质量多步推理能力压缩为可实时执行的轻量模型。相对已有方法的增量边界在于:从“依赖大模型在线推理”转向“离线蒸馏后边缘高效推理”。

可验证创新产出包括:可复现实验协议、系统原型与关键模块实现、阶段论文与专利成果。


\section{预期成果}

本项目将构建多智能体协同的空间感知图像恢复系统,在理论与方法上形成“空间退化图表示—层级协同规划—轨迹蒸馏部署”的完整技术链。

预期产生的学术成果如下:

(1)形成空间退化分析算法、评测脚本与实验数据资产,完成阶段论文投稿。

(2)形成层级多智能体恢复原型系统与调度策略库,申请发明专利。

(3)形成轻量路由部署模型与端侧测试报告,完成系统化验证。

(4)在重要期刊或会议发表论文2--4篇,申请发明专利1项,培养硕士研究生1--2名。

\section{参考文献}

{\fontsize{9pt}{12pt}\selectfont
\printbibliography[heading=none]
}


\setlength{\parindent}{0pt}%
\end{proposalboxthick}

% ==============================================
%  第二部分:前期研究基础
% ==============================================

% \vspace{4em}
\newpage

% 标题:黑体 四号(14pt)
\noindent{\fontsize{14pt}{14pt}\selectfont\sffamily 四、前期研究基础\par}%
\vspace{1em}
\nointerlineskip\vspace{4pt}%

% --- 可分页边框区:说明 + 填写区域(粗边框) ---
\begin{proposalboxthick}

% === 指导说明 ===
{\fontsize{10.5pt}{15pt}\selectfont\bfseries
本表参照以下提纲撰写,1000字以内。\par}%
\vspace{4pt}%
{\fontsize{10.5pt}{15pt}\selectfont
1.~课题组近五年来已有的相关研究成果(负责人和参加者分开填写)。%
\par
2.~课题负责人近五年来曾完成哪些重要研究课题,科研成果的社会评价(引用、转载、获奖及被采纳情况)。%
\par
3.~为本课题研究已作的前期准备工作(已收集的数据,进行的调查研究等)。%
\par
4.~本课题与已立项项目、博士论文(博士后出站报告)的联系与区别。%
\par\vspace{6pt}%
}%


% ========================================
% 在此填写内容
% ========================================

\setlength{\parindent}{2em}%
\setcounter{section}{0}
\setcounter{subsection}{0}

\section{课题负责人近五年来已有的相关研究成果}

课题负责人近五年持续围绕计算机视觉与遥感智能处理开展研究,研究方向主要包括:图像恢复、连续比例因子超分辨率、视觉语言多模态学习、基于扩散模型的生成式方法、遥感变化分析等。相关成果发表于IEEE Transactions on Geoscience and Remote Sensing、IEEE Geoscience and Remote Sensing Letters、Remote Sensing、ICASSP、IGARSS等期刊与会议,研究主题覆盖:盲超分辨率重建、连续比例因子超分辨率、跨模态检索、变化描述与变化问答等。代表性成果可归纳为三类。

\textbf{一、连续比例因子超分辨率建模:}针对固定比例模型训练与存储成本高、且对非整数倍率泛化不足的问题,构建了连续尺度感知建模框架,引入跨尺度特征耦合与动态上采样机制,实现了单模型覆盖连续倍率重建并兼顾质量与效率,相关工作发表于IEEE TGRS、IGARSS等期刊与会议,为“区域工具参数连续优化”提供直接方法学基础。

\textbf{二、复杂退化过程下的生成式恢复:}针对真实场景退化分布复杂、传统单一退化假设鲁棒性不足的问题,结合生成式建模与退化过程表征,提升模型对复杂退化的适应能力,增强复杂场景中的恢复稳定性与视觉一致性,相关工作发表于IEEE TGRS等期刊,为“空间退化图驱动的策略规划”提供可迁移技术储备。

\textbf{三、遥感多模态理解与指令驱动分析:}针对仅依赖视觉特征难以支撑高层语义决策的问题,开展跨模态检索、变化描述与变化问答研究,探索指令驱动视觉分析流程,相关工作发表于IEEE ICASSP、Remote Sensing等会议或期刊,为“多智能体协同推理与反馈”提供实现基础。

上述成果中的“条件随机标准化流”模型被武汉大学张良陪教授团队引用并给予高度评价;“连续尺度超分辨率”工作多次被顶级会议AAAI、CVPR的研究引用与评价;“交互式变化分析框架”与数据集ChangeChat-87k被北京航空航天大学史振威教授团队的最新综述引用,并给予了详细介绍与积极评价。
\section{课题负责人近五年的项目经历}

近五年主持/参与项目清单如下(来源于个人主页公开信息):

(1)2025.01--2027.12,国家自然科学基金青年科学基金项目:开放场景下认知启发的遥感影像超分辨率重建方法研究,主持。

(2)2024.06--2027.06,北京外国语大学学术青年创新团队项目:生成式大语言模型的核心价值观对齐研究,参与。

(3)2024.06--2027.06,北京外国语大学项目:基于状态空间扩散模型的遥感影像变化描述方法研究,参与。

(4)2023.01--2025.12,国家自然科学基金面上项目:演进学习框架下协同感知显著性引导的弱标注遥感影像语义分割方法研究,参与。

(5)2022.09--2025.09,北京外国语大学项目:自适应学习框架下显著性引导的遥感影像超分辨率重建方法研究,主持。

上述项目对本课题形成的能力支撑包括:  
1)\textbf{数据与实验支撑}:沉淀了多类遥感图像处理任务的数据处理与评测流程;  
2)\textbf{方法与实现支撑}:积累了超分辨率、生成式建模与多模态分析的实现经验;  
3)\textbf{协同与组织支撑}:形成了跨任务协同研究与阶段成果产出的组织基础。


\section{前期准备工作}

\subsection{理论与技术储备}

(1)图像恢复与超分辨率方法学储备:系统梳理盲超分辨率重建、连续比例因子建模的理论框架,研究深度降质先验(Deep Degradation Prior)与在线学习机制在遥感影像复原中的应用,掌握跨尺度特征耦合与动态上采样方法,为区域工具参数连续优化提供理论基础。

(2)多模态遥感理解储备:深入分析视觉语言模型(VLM)在遥感场景理解中的适配策略,研究跨模态检索、变化描述与变化问答任务的技术路线,探索指令驱动视觉分析流程,为多智能体协同推理与反馈机制提供技术参考。

(3)生成式建模储备:研究标准化流(Normalizing Flow)、扩散模型(Diffusion Model)与隐式神经表示(INR)在复杂退化分布建模中的优势,分析生成式方法在真实场景退化恢复中的适用性,为空间退化图驱动的策略规划提供可迁移技术方案。

\subsection{数据与实验环境构建}

(1)遥感图像恢复实验环境搭建:配置基于PyTorch的统一训练与验证框架,集成主流超分辨率算法(如ESRGAN、Real-ESRGAN、SwinIR)的复现流程,建立标准化的数据预处理与评测指标计算模块,确保实验可稳定复现。

(2)多模态任务实验流程建设:构建跨模态任务的端到端实验管线,涵盖数据加载、特征提取、联合训练与结果可视化环节,形成可复用的数据处理与评测流程。

(3)多源遥感数据集整理:收集公开非配对遥感数据集(如Sentinel-2、Landsat、WorldView),构建低分辨率(LR)与高分辨率(HR)非配对样本库,涵盖不同传感器、光照条件与地物类型;开发数据预处理工具链(辐射校正、去云、配准),确保数据分布多样性。

\subsection{关键技术预研}

(1)连续比例因子恢复预研:针对效率与质量平衡问题,研究轻量化模型架构与动态参数策略,验证隐式神经表示在非整数倍率(如$\times$2.5、$\times$3.5)重建中的可行性,初步结果支持该技术路径的有效性。后续计划纳入区域级参数调优机制。

(2)指令驱动决策预研:针对复杂场景下的决策稳定性问题,研究视觉语言联合分析与反馈闭环机制,验证自然语言指令在遥感影像分析任务中的引导效果,初步确认指令驱动方法可有效提升分析准确性。后续计划扩展至多智能体交互流程。

(3)复杂退化建模预研:针对退化建模与部署效率的矛盾,研究生成式建模结合轨迹蒸馏的技术路线,验证潜空间建模在复杂退化场景下的鲁棒性,初步结果表明该方法可有效处理真实场景退化分布。后续计划推进轻量模型蒸馏与边缘部署测试。

(4)风险识别与改进策略:针对预研中可能出现的训练不稳定、跨场景泛化不足等问题,已制定困难样本增强、分阶段训练与多指标联合评估策略。


\section{本课题与已立项项目的联系与区别}

已立项项目主要包括:国家自然科学基金青年项目“开放场景下认知启发的遥感影像超分辨率重建方法研究”、新教师科研启动项目“自适应学习框架下显著性引导的遥感影像超分辨率重建方法研究”。

(1)联系:本课题与既有项目均聚焦遥感图像增强与理解,均强调利用前沿人工智能方法提升模型在复杂场景下的性能与泛化能力。

(2)区别:  
针对国家自然科学基金青年项目:  
核心:强调“认知启发”与开放场景推理机制。  
区别:本课题不以认知机制建模为主线,而以“空间退化状态驱动的层级多智能体决策”作为核心增量。  
针对新教师科研启动项目:  
核心:强调“显著性引导”的局部增强策略。  
区别:本课题从局部增强扩展为“区域感知—工具路由—参数优化—蒸馏部署”的全流程系统方案。

(3)延续性:本课题充分继承既有项目在超分辨率、生成式建模与多模态理解方面的积累,在此基础上推进跨模块协同与轻量化部署,具有明确连续性。

(4)重复性规避说明:本课题与既有项目在研究对象、技术路线、评价指标和预期输出上均有清晰区分:从单模型性能提升转向系统级决策能力提升,重点考察复杂退化场景下的综合质量—效率指标。


\setlength{\parindent}{0pt}%
\end{proposalboxthick}

\end{document}
