% 1.1 立项依据 - 基于多智能体协同的空间感知图像恢复
% 生成时间: 2026-02-27
% 结构: 4段闭环 (价值/现状/科学问题/切入点)

\subsubsection{研究背景}

图像恢复旨在从受损的观测图像中重建高质量的清晰图像,是计算机视觉领域的基础性研究课题。在安防监控、自动驾驶、医学成像、遥感观测等实际应用中,图像在采集与传输过程中不可避免地受到噪声、模糊、雾霾、雨线、低光照、压缩伪影等多种退化因素的影响,严重制约了后续视觉分析任务的精度与可靠性。

当前图像恢复研究面临的核心矛盾在于:真实场景中的退化呈现显著的空间异质性——同一幅图像的不同区域往往同时受到类型迥异、程度各异的退化影响,而现有方法普遍采用全图统一的处理策略,无法根据区域级退化特征进行差异化恢复。在预研复现实验中,代表性模型在复杂混合退化场景下的 PSNR 较单一退化测试下降约 1.5--3.0~dB,视觉质量出现稳定退化,且推理开销显著增加。因此,研究具备空间感知能力的精细化图像恢复方法,对提升视觉信息质量、突破现有方法的理论瓶颈具有重要意义。

从学术价值看,推动图像恢复从"全图单策略处理"向"区域级动态决策处理"演进,可丰富复杂退化图像恢复的理论内涵;从方法价值看,促进空间感知、智能体规划与知识蒸馏的深度融合,可形成兼顾恢复质量与计算效率的新型方法体系;从应用价值看,面向安防巡检、智能交通、遥感解译等场景提供高可靠视觉基础能力,可服务智能感知系统的工程落地。

\subsubsection{国内外研究现状}

图像恢复领域近十年经历了从卷积神经网络到 Transformer、从单任务专用模型到统一化 All-in-One(AiO)框架、从确定性回归方法到扩散生成模型、再到基于智能体的动态决策范式的多次技术跃迁。以下从四个维度梳理代表性进展及其理论局限。

\textbf{(1)基于 Transformer 的单任务方法}

SwinIR\cite{swinir} 通过移位窗口多头自注意力(SW-MSA)将计算复杂度降至线性,在去噪、超分辨率等任务上建立了基准性能。Restormer\cite{restormer} 提出转置注意力机制,沿通道维度计算注意力以保持全局感受野,在去噪基准上取得当时的最优结果。后续工作从频域计算(FFTformer\cite{fftformer})、状态空间模型(VmambaIR\cite{vmambair})等角度进一步提升效率。然而,这些方法的根本局限在于:需为每种退化类型单独维护专用模型,面对多退化共存的复杂场景时难以有效应对;且全图统一的处理策略完全未能利用退化的空间分布信息。

\textbf{(2)All-in-One 统一恢复方法}

AiO 方法旨在以单一模型处理多种退化类型。AirNet\cite{airnet} 通过对比式降质编码器构建退化判别表征,实现无需显式退化标签的自适应处理。PromptIR\cite{promptir} 引入可学习提示向量对 Transformer 主干进行软条件化,在去噪、去雾、去雨、去模糊四类任务内实现统一恢复。InstructIR\cite{instructir} 首次将自然语言理解引入图像恢复,以用户指令控制恢复过程。然而,Jiang 等人\cite{aio_survey} 的系统综述指出,AiO 方法存在三项持续局限:性能受训练集退化类型的强约束,泛化能力不足;全图单一处理流程从根本上忽视退化的空间异质性;单次前向推理的静态模式无法基于中间结果进行迭代细化。

\textbf{(3)基于扩散模型的生成式方法}

扩散概率模型通过迭代去噪过程建模复杂数据分布,在感知质量指标上取得显著优势。DiffBIR\cite{diffbir} 开创了两阶段范式,利用预训练 Stable Diffusion 的生成先验进行高质量细节增强。然而,扩散模型面临迭代采样的高计算成本制约,且强大的生成先验可能引入与原始场景不符的"幻觉细节",在高保真度应用中风险较大。

\textbf{(4)基于智能体的动态决策方法}

AgenticIR\cite{agenticir} 建立了五阶段动态规划框架(感知-调度-执行-反思-重调度),将恢复过程重新定义为具有闭环反馈的智能决策问题。MAIR\cite{mair} 引入多智能体协作架构,利用退化先验知识压缩工具选择空间,将推理效率提升 44\%。Q-Agent\cite{qagent} 以预期质量提升作为贪婪选择准则,将规划复杂度降至线性。然而,上述方法均将整幅图像视为单一实体处理,未能在决策框架中纳入退化的空间分布信息,导致在异质退化场景下出现局部过度处理或处理不足的问题。

\subsubsection{现有研究的局限性}

综合上述分析,现有研究存在以下三个层面的理论局限,构成本项目的直接研究动因。

\textbf{(1)退化表征层面:空间异质性建模不足}

现有方法对退化的表征停留在图像级或全局特征级,缺乏对"退化类型—严重程度—空间位置"耦合关系的区域级建模能力。Transformer 类方法虽具备全局上下文建模能力,但其窗口划分策略与退化区域边界往往不对齐,导致在窗口边界引入分割伪影;AiO 方法的提示向量虽能区分不同退化类型,但无法定位退化在空间上的具体分布;基于智能体的方法虽引入动态决策,但感知阶段缺乏对空间区域的精细化分析能力。这一表征粒度的缺失,使得现有方法在面对异质退化时难以实现区域级的差异化处理。

\textbf{(2)恢复决策层面:缺乏工具链与参数的联合优化机制}

现有基于智能体的方法在决策层面存在两大瓶颈:一是工具选择策略多基于贪婪搜索或经验规则,缺乏对"工具链组合—超参数配置—区域特征"的联合优化能力;二是规划与执行之间存在语义鸿沟,协调智能体生成的抽象计划难以精确映射到具体工具的参数配置。这导致恢复策略的选择往往次优,且无法根据区域级退化特征进行自适应调整。

\textbf{(3)部署效率层面:大模型依赖导致的实时性瓶颈}

现有智能体化方法普遍依赖 VLM/LLM 进行感知与规划,虽然增强了决策灵活性,但大模型的推理时延和算力消耗显著制约了实时部署可行性。如何在保持空间感知与动态决策能力的前提下,通过知识蒸馏、轨迹压缩等手段降低推理开销,是从研究原型走向实用部署的关键挑战。

基于上述瓶颈,本项目凝练出以下关键科学问题:

\textbf{科学问题一}:如何构建融合语义分割先验与退化特征分析的区域级退化表征机制,实现"退化类型—严重程度—空间边界"的联合建模?

\textbf{科学问题二}:如何设计面向区域级恢复任务的动态规划与工具链优化方法,实现自适应于异质退化分布的差异化恢复策略?

\textbf{科学问题三}:如何建立从大模型规划轨迹到小模型推理行为的有效蒸馏机制,在保持恢复质量的同时实现高效部署?

对应上述科学问题,本项目的核心科学假设为:\textbf{通过构建"空间退化感知—多智能体协同—规划轨迹蒸馏"的层级化框架,可以实现对复杂异质退化图像的精细化、高效化恢复,其恢复质量优于全图统一处理策略,且推理效率显著优于基于大模型的动态规划方法。}

\subsubsection{研究切入点}

针对现有研究在空间感知能力方面的缺失,本项目拟研究多智能体协同的空间感知图像恢复方法,形成"可感知、可决策、可迭代、可部署"的新型恢复范式。本项目相对现有工作的差异化切入点体现在以下三方面:

\textbf{(1)表征创新:空间退化图的区域级建模}

不同于现有方法的全局表征策略,本项目提出"空间退化图"(Spatial Degradation Map, SDM)概念,将退化表征从图像级细化为区域级。通过融合 SAM 的语义分割能力与退化感知网络的特征提取能力,构建显式编码"退化类型—严重程度—空间边界"的三维张量表征,为差异化恢复提供精细化的输入条件。

\textbf{(2)方法创新:层级化多智能体协同架构}

针对异质退化场景,本项目提出"协调智能体—区域子智能体"的层级化协作架构。协调智能体基于空间退化图进行全局规划,将图像分解为具有同质退化特征的区域子任务;各区域子智能体针对特定退化类型执行专用恢复工具链;最后通过边界感知融合模块整合各区域结果,实现全局一致的精细化恢复。

\textbf{(3)机制创新:规划轨迹蒸馏的效率优化}

为解决大模型依赖导致的部署瓶颈,本项目提出"规划轨迹蒸馏"机制:在训练阶段利用大模型的强大规划能力生成高质量决策轨迹,通过行为克隆与策略蒸馏将规划知识迁移至轻量级小模型,在保持空间感知与动态决策能力的同时实现实时推理。

综上所述,本项目通过空间感知表征、多智能体协同与轨迹蒸馏的有机结合,有望突破现有方法在处理复杂异质退化方面的理论瓶颈,为图像恢复领域提供新的研究思路与技术路径。后续研究内容将围绕上述三个层面的创新点展开系统深入的研究。
