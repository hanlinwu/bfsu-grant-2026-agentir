\section{创新之处}

\textbf{(1)理论创新:空间退化状态驱动的恢复决策框架。}传统图像恢复方法将整幅图像视为单一退化实体,采用统一恢复函数进行端到端映射,忽视了真实场景中退化的空间异质性。本项目提出以空间退化图作为决策先验,将恢复问题重新定义为状态条件策略函数,使恢复策略能够根据区域退化状态动态调整,由“统一恢复函数”转向“状态条件策略函数”,为复杂退化场景下的精细化恢复提供理论支撑。

\textbf{(2)方法创新:全局调度器与区域专家协同的层级多智能体机制。}现有基于智能体的恢复方法采用整图级统一规划,未能利用空间分布信息。本项目提出“全局调度器+区域专家”的两层协同架构:全局层依据退化先验分配处理优先级,区域层执行工具链选择与连续参数优化,并通过质量反馈迭代更新。该机制突破了离散工具选择与连续参数调节的分离局限,实现了“离散路由+连续调参”的联合优化,从全图统一决策扩展到区域级差异化决策。

\textbf{(3)技术创新:基于轨迹蒸馏的轻量路由部署方案。}基于大模型的智能体系统在线推理成本高、部署门槛高,难以满足边缘实时应用需求。本项目提出离线轨迹收集与蒸馏训练相结合的技术路线:利用多智能体系统在合成数据上离线运行收集轨迹数据,训练轻量级路由网络直接预测区域分割、退化类型、工具调度与参数配置,实现端到端推理。部署时无需依赖大模型在线调用,仅需轻量网络与恢复工具库即可完成实时恢复,在保持恢复质量的同时显著降低部署时延与算力开销。