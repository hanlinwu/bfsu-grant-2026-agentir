\section{创新之处}

本项目创新点按“理论创新—方法创新—技术创新”组织如下:

一是\textbf{理论创新}:提出“空间退化状态驱动的恢复决策”框架,将图像恢复问题从全图单步映射扩展为区域状态感知下的多步决策问题。相对已有方法的增量边界在于:由“统一恢复函数”转向“状态条件策略函数”。

二是\textbf{方法创新}:提出全局调度器与区域专家协同的层级多智能体机制,实现工具链选择与连续参数优化一体化。相对已有方法的增量边界在于:从离散工具选择扩展到“离散路由 + 连续调参”的联合优化。

三是\textbf{技术创新}:提出基于轨迹蒸馏的轻量路由部署方案,将高质量多步推理能力压缩为可实时执行的轻量模型。相对已有方法的增量边界在于:从“依赖大模型在线推理”转向“离线蒸馏后边缘高效推理”。

可验证创新产出包括:可复现实验协议、系统原型与关键模块实现、阶段论文与专利成果。

