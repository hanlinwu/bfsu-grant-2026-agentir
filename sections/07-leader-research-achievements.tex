\section{课题负责人近五年来已有的相关研究成果}

课题负责人近五年持续围绕计算机视觉与遥感智能处理开展研究,研究方向主要包括:图像恢复、连续比例因子超分辨率、视觉语言多模态学习、基于扩散模型的生成式方法、遥感变化分析等。相关成果发表于IEEE Transactions on Geoscience and Remote Sensing、IEEE Geoscience and Remote Sensing Letters、Remote Sensing、ICASSP、IGARSS等期刊与会议,研究主题覆盖:盲超分辨率重建、连续比例因子超分辨率、跨模态检索、变化描述与变化问答等。代表性成果可归纳为三类。

\textbf{一、连续比例因子超分辨率建模:}针对固定比例模型训练与存储成本高、且对非整数倍率泛化不足的问题,构建了连续尺度感知建模框架,引入跨尺度特征耦合与动态上采样机制,实现了单模型覆盖连续倍率重建并兼顾质量与效率,相关工作发表于IEEE TGRS、IGARSS等期刊与会议,为“区域工具参数连续优化”提供直接方法学基础。

\textbf{二、复杂退化过程下的生成式恢复:}针对真实场景退化分布复杂、传统单一退化假设鲁棒性不足的问题,结合生成式建模与退化过程表征,提升模型对复杂退化的适应能力,增强复杂场景中的恢复稳定性与视觉一致性,相关工作发表于IEEE TGRS等期刊,为“空间退化图驱动的策略规划”提供可迁移技术储备。

\textbf{三、遥感多模态理解与指令驱动分析:}针对仅依赖视觉特征难以支撑高层语义决策的问题,开展跨模态检索、变化描述与变化问答研究,探索指令驱动视觉分析流程,相关工作发表于IEEE ICASSP、Remote Sensing等会议或期刊,为“多智能体协同推理与反馈”提供实现基础。

上述成果中的“条件随机标准化流”模型被武汉大学张良陪教授团队引用并给予高度评价;“连续尺度超分辨率”工作多次被顶级会议AAAI、CVPR的研究引用与评价;“交互式变化分析框架”与数据集ChangeChat-87k被北京航空航天大学史振威教授团队的最新综述引用,并给予了详细介绍与积极评价。