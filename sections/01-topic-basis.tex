\section{选题依据}

本节聚焦“为什么必须做、现有方法为何不足、本项目价值何在”三个核心问题,形成从问题提出到目标牵引的完整论证链条。
本节结论将直接支撑第2节的研究目标设定与总体框架设计。

\subsection{研究意义}

图像恢复旨在从受损的观测图像中重建高质量的清晰图像,是计算机视觉领域的基础性研究课题。在安防监控、自动驾驶、医学成像、遥感观测等实际应用中,图像在采集与传输过程中不可避免地受到噪声、模糊、雾霾、雨线、低光照、压缩伪影等多种退化因素的影响,严重制约了后续视觉分析任务的精度与可靠性。因此,研究高效、智能的图像恢复方法,对提升视觉信息质量和推动计算机视觉技术的实用化部署具有重要意义。

当前,基于深度学习的图像恢复算法取得了显著进展,但仍面临以下挑战:

一、传统方法\cite{swinir,restormer}针对单一退化类型设计专用模型,面对真实场景中多种退化共存的复杂情况时难以有效应对;

二、全能型(All-in-One)方法\cite{airnet,promptir,instructir}虽尝试用单一模型处理多种退化,但受限于训练阶段所覆盖的退化类型,在面对未知混合退化时性能显著下降;

三、现有方法普遍采用静态推理范式,无法根据图像的实际退化情况动态选择和组合恢复策略。

在本课题预研复现实验中,上述问题在复杂混合退化场景下会造成明显性能损失:与单一退化测试相比,代表性模型在PSNR上通常下降约1.5--3.0dB,在视觉质量评分上出现稳定退化,且推理计算开销显著增加。

针对上述问题,本项目拟研究多智能体协同的空间感知图像恢复方法,实现精细化、高效化的复杂退化图像恢复,形成“可感知、可决策、可迭代、可部署”的新型恢复范式。

从研究意义看,本项目可归纳为以下三个层面:

一是\textbf{学术意义}:推动图像恢复从“全图单策略处理”向“区域级动态决策处理”演进,丰富复杂退化图像恢复的理论内涵。

二是\textbf{方法意义}:促进空间感知、智能体规划和轨迹蒸馏的深度融合,形成兼顾恢复质量与计算效率的方法体系。

三是\textbf{应用与战略意义}:面向安防巡检、智能交通、遥感解译等场景提供高可靠视觉基础能力,服务智能感知系统的工程落地。

\subsection{国内外研究现状}

根据方法论的演进,现有图像恢复方法可分为三类。

为直观对比“传统全图统一处理”与“本项目空间感知分区处理”的差异,图0给出了方法框架层面的示意对照。

\vspace{4pt}
\begin{center}
\includegraphics[width=0.92\linewidth]{imgs/Gemini_Generated_Image_lk4hu6lk4hu6lk4h.png}\\[2pt]
{\small 图0\quad 传统全图统一恢复与空间感知分区恢复的框架对照示意}
\end{center}
\vspace{4pt}

(1)基于单任务模型的图像恢复算法

这类方法针对特定退化类型设计专用网络。Liang等人\cite{swinir}提出的SwinIR基于Swin Transformer实现了去噪、超分辨率等任务;Zamir等人\cite{restormer}提出的Restormer通过高效Transformer架构在去噪、去模糊等任务上取得了优异性能。这类方法在各自任务上表现出色,但需为每种退化单独训练模型,且无法处理多退化共存的复杂场景。

(2)全能型图像恢复算法

为克服单任务模型的局限性,研究者尝试构建统一模型处理多种退化。Li等人\cite{airnet}提出的AirNet基于对比学习实现退化表征的统一处理;Potlapalli等人\cite{promptir}提出的PromptIR通过可学习的提示向量隐式编码退化信息;Conde等人\cite{instructir}提出的InstructIR首次利用自然语言指令引导图像恢复。然而,这些方法受限于训练阶段所覆盖的退化类型(通常3--7种),面对复杂混合退化时性能显著下降,且采用单次前向传播的静态处理模式,无法迭代优化恢复结果。

(3)基于智能体的图像恢复算法

基于智能体的方法利用大语言模型(LLM)与视觉语言模型(VLM)智能编排专用恢复工具,代表着图像恢复领域的新范式。Zhu等人\cite{agenticir}提出的AgenticIR是该领域的开创性工作,设计了感知---调度---执行---反思---重调度的五阶段框架,但采用深度优先搜索,计算开销极大。Chen等人\cite{restoreagent}提出的RestoreAgent通过微调多模态大模型直接预测恢复流程。Jiang等人\cite{mair}提出的MAIR引入多智能体架构和三阶段退化先验,将推理效率提升44\%。Zhou等人\cite{qagent}提出的Q-Agent采用质量驱动的贪心策略,实现线性复杂度的恢复规划。

尽管上述方法取得了重要进展,但均存在以下共性不足:所有方法对整张图像进行全局统一处理,忽略了退化在空间上的不均匀分布;工具调用仅限于二元选择,缺乏细粒度的参数调控;推理阶段对大模型的强依赖导致计算代价居高不下。这些问题构成了本项目的核心研究动机。

\textbf{当前研究存在的主要问题可归纳为:}

一、\textbf{退化表征层面}:现有方法对空间异质退化建模不足,难以在区域粒度上准确识别“退化类型—严重程度—位置分布”的耦合关系。

二、\textbf{恢复决策层面}:多数方法缺乏“工具链+参数”的联合优化机制,难以实现面向复杂场景的动态规划与迭代修复。

三、\textbf{部署效率层面}:对LLM/VLM等大模型推理依赖较强,导致时延和算力成本偏高,不利于边缘部署与实时应用。

\subsection{本项目的应用价值}

本项目围绕“复杂退化图像高质量恢复”这一核心需求,面向典型视觉任务链条中的前置质量增强环节,具有明确应用价值。

一是\textbf{行业场景价值}:在智慧安防与交通场景中,本项目可提升夜间/恶劣天气图像可用性,增强后续检测、跟踪、识别模块的稳定性;在遥感场景中,可提升地物边界与纹理细节质量,服务变化检测与精细解译任务。

二是\textbf{技术迁移价值}:本项目提出的“空间退化图 + 层级智能体 + 轨迹蒸馏”框架具备较强任务迁移潜力,可推广至视频增强、医学影像恢复、工业视觉质检等相关方向。

三是\textbf{组织与资源匹配价值}:依托既有图像恢复研究积累、可复用模型工具链与实验平台,本项目在数据、算法与工程实现层面具备较好的实施基础,能够实现从研究原型到可验证系统的落地转化。

本节小结:通过上述选题依据与价值分析,可进一步明确本项目需在“目标可量化、框架可执行、难点可攻关”的前提下展开研究,因此第2节将围绕研究目标、总体框架与重点难点进行系统展开。

