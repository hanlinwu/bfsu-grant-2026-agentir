\section{本课题与已立项项目的联系与区别}

已立项项目主要包括:国家自然科学基金青年项目“开放场景下认知启发的遥感影像超分辨率重建方法研究”、新教师科研启动项目“自适应学习框架下显著性引导的遥感影像超分辨率重建方法研究”。

(1)联系:本课题与既有项目均聚焦遥感图像增强与理解,均强调利用前沿人工智能方法提升模型在复杂场景下的性能与泛化能力。

(2)区别:  
针对国家自然科学基金青年项目:  
核心:强调“认知启发”与开放场景推理机制。  
区别:本课题不以认知机制建模为主线,而以“空间退化状态驱动的层级多智能体决策”作为核心增量。  
针对新教师科研启动项目:  
核心:强调“显著性引导”的局部增强策略。  
区别:本课题从局部增强扩展为“区域感知—工具路由—参数优化—蒸馏部署”的全流程系统方案。

(3)延续性:本课题充分继承既有项目在超分辨率、生成式建模与多模态理解方面的积累,在此基础上推进跨模块协同与轻量化部署,具有明确连续性。

(4)重复性规避说明:本课题与既有项目在研究对象、技术路线、评价指标和预期输出上均有清晰区分:从单模型性能提升转向系统级决策能力提升,重点考察复杂退化场景下的综合质量—效率指标。

