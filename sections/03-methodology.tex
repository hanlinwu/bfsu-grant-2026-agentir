\section{思路方法}

本节围绕“如何做成”展开,按照技术流程、实施计划与可行性支撑三层推进,重点回答模型如何训练、如何推理、如何规避风险。

\subsection{具体研究方法}

本项目的整体研究思路为"感知$\to$规划$\to$执行$\to$蒸馏"。首先构建空间感知模块实现区域级退化分析,然后设计层级智能体进行协同规划与执行,最后通过轨迹蒸馏实现轻量化部署。三个研究内容层层递进、相互依托。具体方法流程如图2所示。

% === 图2 占位符 ===
% 【画图 AI Prompt(详细描述)】:
% 绘制一张学术论文风格的详细方法流程图,展示"多智能体协同的空间感知图像恢复"的
% 完整技术流程。白色背景,专业扁平化设计,整体布局为从上到下的三行结构,
% 分别对应三个研究内容,行与行之间有清晰的数据流箭头连接。
%
% 【第一行:空间感知退化分析(浅蓝色背景带)】
%   从左到右排列:
%   (a) 输入:一张混合退化图像 y(城市街景,不同区域有不同退化:
%       天空区域有雾霾、建筑区域有噪声、道路区域有模糊、植被区域有雨线)。
%   (b) SAM分割步骤:图像被SAM模型分割为4-5个语义区域,
%       每个区域用不同颜色的半透明掩码覆盖(天空=蓝色掩码、建筑=橙色掩码、
%       道路=灰色掩码、植被=绿色掩码),区域边界用白色线条标出。
%       上方标注公式 "{R_i} = SAM(y)"。
%   (c) VLM逐区域分析步骤:展示为多个并行的分析通道,每个通道对应一个区域。
%       每个通道显示:区域裁剪图 → DepictQA模型图标 → 结构化输出。
%       用文字气泡展示分析结果示例:
%       "R1(天空): 雾霾=0.8, 噪声=0.1, 模糊=0.0"
%       "R2(建筑): 噪声=0.9, 雾霾=0.2, 模糊=0.1"
%       "R3(道路): 模糊=0.7, 噪声=0.3, 雾霾=0.0"
%       上方标注 "(d_i, s_i) = VLM(y_{R_i})"。
%   (d) 输出:空间退化图 M,以彩色热力图叠加在原图上展示,
%       每个区域标注其主要退化类型和严重程度数值。
%       旁边用数学符号标注 "M = {(R_i, d_i, s_i)}"。
%
% 【第二行:层级多智能体协同规划(浅橙色背景带)】
%   中央为双层架构示意:
%   (上层——全局调度器) 一个大的圆角矩形节点,内部显示:
%       左侧:接收空间退化图 M 作为输入(从第一行有粗箭头连下来)。
%       中部:退化先验排序规则的可视化——三个箭头从右到左排列,
%       分别标注"1.压缩退化(JPEG伪影)""2.成像退化(噪声/模糊)""3.场景退化(雾霾/雨线)",
%       表示逆序处理优先级。
%       右侧:输出各区域的处理优先级队列和策略分配。
%   (下层——区域专家组) 3个并排的小圆角矩形,分别标注"区域专家1""区域专家2""区域专家3"。
%       每个专家内部显示:
%       - 工具选择:从一个竖排的"工具库"(包含图标:Denoiser去噪、Deblurrer去模糊、
%         Dehazer去雾、Derainer去雨、SR超分辨率、JPEG去伪影)中选出2-3个工具,
%         用箭头连成工具链,例如 Expert1: Dehazer→Denoiser。
%       - 参数设定:每个选中工具旁有一个小滑块/旋钮,标注连续参数值
%         如 "σ=0.7""scale=2×"。
%       - 工具链右侧显示修复后的区域图像小缩略图。
%   (反馈回路) 每个区域专家的输出有箭头指向一个"质量评估 Q(·)"模块
%       (DepictQA图标),评估模块输出分数(如"Q=0.82"),
%       与阈值 τ 比较:若 Q < τ 则红色箭头反馈回区域专家(标注"调参重试"),
%       若 Q ≥ τ 则绿色箭头向右输出(标注"通过")。
%   (融合输出) 所有通过质量阈值的区域修复结果汇聚到一个"掩码加权融合 ⊕"节点,
%       输出最终恢复图像 x̂。公式标注 "x̂ = Σ R_i ⊙ x̂_{R_i}"。
%
% 【第三行:轨迹蒸馏与轻量部署(浅绿色背景带)】
%   左侧——数据收集:
%       从第二行有一条大的虚线箭头标注"离线运行 & 轨迹记录",
%       指向多组轨迹数据的示意:3-4个小卡片,每个卡片左侧是退化图像缩略图,
%       右侧是该图像对应的决策轨迹(区域划分 + 工具链 + 参数),
%       用文字简写如 "y→{R1:Dehaze(0.8)→Denoise(0.5), R2:Deblur(0.6)→SR(2×)}"。
%       标注 "轨迹数据集 T, |T|=M"。
%   中部——蒸馏训练:
%       一个"知识蒸馏"箭头(粗虚线箭头,标注"监督学习")从轨迹数据指向
%       一个轻量级网络结构示意图。该网络以退化图像为输入(左侧),
%       经过一个小型编码器(几层卷积块),输出四个分支头:
%       "分割头"(输出区域掩码)、"分类头"(输出退化类型)、
%       "路由头"(输出工具选择)、"参数头"(输出连续参数)。
%       网络标注为 "Φ_ω"。损失函数标注在网络下方:
%       "L = L_seg + λ1·L_cls + λ2·L_route + λ3·L_param"。
%   右侧——部署推理:
%       轻量路由网络 Φ_ω 直接连接工具库执行恢复,输出最终图像。
%       旁边标注关键优势:"无需LLM/VLM"(旁有一个被划掉的大模型图标)、
%       "实时推理"(旁有一个时钟/闪电图标)。
%
% 【全局标注】:
%   三行左侧分别标注研究内容编号 "(1)""(2)""(3)"。
%   行与行之间的连接箭头标注传递的数据名称。
%   整体配色:蓝色系(感知)、橙色系(规划)、绿色系(部署),
%   与图1的三模块颜色保持一致。
%
\vspace{6pt}
\begin{center}
\includegraphics[width=\linewidth]{imgs/Gemini_Generated_Image_qjhpc4qjhpc4qjhp.png}\\[2pt]
{\small 图2\quad 具体研究方法流程}
\end{center}
\vspace{6pt}

(1)空间感知退化分析

给定退化图像$\mathbf{y} \in \mathbb{R}^{H \times W \times 3}$,本模块旨在生成空间退化图$\mathcal{M}$,为后续智能体规划提供结构化输入。

步骤一:利用零样本分割模型SAM\cite{sam}将图像划分为$N$个语义一致的区域:
\begin{equation}
  \{R_i\}_{i=1}^{N} = \mathrm{SAM}(\mathbf{y}), \quad \bigcup\nolimits_{i=1}^{N} R_i = \Omega, \quad R_i \cap R_j = \emptyset \ (i \neq j)
\end{equation}
其中$\Omega$为图像像素域。步骤二:对每个区域$R_i$,将其裁剪区域$\mathbf{y}_{R_i}$输入微调的视觉语言模型DepictQA\cite{depictqa},采用链式思维推理逐类判断退化是否存在及其程度,输出退化类型向量$\mathbf{d}_i \in \{0,1\}^{C}$与对应严重程度向量$\mathbf{s}_i \in [0,1]^{C}$,其中$C$为预定义退化类别总数。步骤三:汇总所有区域分析结果,构建空间退化图$\mathcal{M} = \{(R_i, \mathbf{d}_i, \mathbf{s}_i)\}_{i=1}^{N}$。

输入输出定义:输入为退化图像$\mathbf{y}$;输出为结构化空间退化图$\mathcal{M}$,其可直接作为后续决策模块的状态描述。

(2)层级多智能体协同规划

基于空间退化图$\mathcal{M}$,设计全局调度器与区域专家的两层协同架构。

步骤一:全局调度器依据退化先验(场景退化$\to$成像退化$\to$压缩退化的逆序处理原则\cite{mair}),为各区域分配处理优先级并制定整体修复策略。步骤二:对于区域$R_i$,区域专家从工具库$\mathcal{F} = \{f_1, f_2, \ldots, f_L\}$中选择恢复工具序列$\pi_i = (f_{\sigma_1}, \ldots, f_{\sigma_{K_i}})$,并为每个工具设定连续参数$\boldsymbol{\theta}_i = (\theta_{\sigma_1}, \ldots, \theta_{\sigma_{K_i}})$,区域修复结果为:
\begin{equation}
  \hat{\mathbf{x}}_{R_i} = (f_{\sigma_{K_i}} \circ \cdots \circ f_{\sigma_1})(\mathbf{y}_{R_i};\, \boldsymbol{\theta}_i)
\end{equation}
步骤三:引入质量评估函数$Q(\cdot)$\cite{depictqa}对修复结果评分,采用贪心策略迭代优化工具参数直至满足质量阈值$\tau$:
\begin{equation}
  (\pi_i^{*}, \boldsymbol{\theta}_i^{*}) = \arg\max_{\pi_i,\, \boldsymbol{\theta}_i} Q(\hat{\mathbf{x}}_{R_i}), \quad \mathrm{s.t.} \quad Q(\hat{\mathbf{x}}_{R_i}) \geq \tau
\end{equation}
步骤四:全局调度器通过区域掩码加权融合各区域结果,合成最终恢复图像$\hat{\mathbf{x}} = \sum_{i=1}^{N} R_i \odot \hat{\mathbf{x}}_{R_i}$,并对全图进行整体质量评估与一致性检查。

输入输出定义:输入为空间退化图$\mathcal{M}$与工具库$\mathcal{F}$;输出为区域最优决策$\{(\pi_i^{*}, \boldsymbol{\theta}_i^{*})\}$与恢复图像$\hat{\mathbf{x}}$。

(3)基于轨迹蒸馏的轻量路由网络

利用上述多智能体系统在大规模合成退化数据集上离线运行,收集决策轨迹数据集$\mathcal{T} = \{(\mathbf{y}^{(j)},\, \mathcal{M}^{(j)},\, \{\pi_i^{*,(j)}, \boldsymbol{\theta}_i^{*,(j)}\}_{i})\}_{j=1}^{M}$。基于该数据集训练轻量级路由网络$\Phi_\omega$,使其直接根据退化图像预测区域划分、退化类型及工具调度方案。训练目标为多任务联合损失:
\begin{equation}
  \mathcal{L} = \mathcal{L}_{\mathrm{seg}} + \lambda_1 \mathcal{L}_{\mathrm{cls}} + \lambda_2 \mathcal{L}_{\mathrm{route}} + \lambda_3 \mathcal{L}_{\mathrm{param}}
\end{equation}
其中$\mathcal{L}_{\mathrm{seg}}$、$\mathcal{L}_{\mathrm{cls}}$、$\mathcal{L}_{\mathrm{route}}$、$\mathcal{L}_{\mathrm{param}}$分别为区域分割、退化分类、工具路由选择与连续参数回归损失,$\lambda_1, \lambda_2, \lambda_3$为平衡系数。部署时仅需轻量路由网络$\Phi_\omega$与恢复工具库,无需依赖LLM/VLM,可实现端侧实时推理。

训练与推理区别:训练阶段使用离线轨迹监督并联合优化多任务损失;推理阶段仅执行前向路由预测与工具调用,避免大模型在线推理开销。

风险与回退策略:针对分割误差传播问题,设置区域置信度门控与候选掩码回退;针对规划误差累积问题,引入全局一致性约束与早停机制;针对蒸馏偏差问题,采用高质量轨迹重加权与困难样本再训练。

\subsection{研究计划}

研究计划按“阶段目标—里程碑—可交付件—验收标准”组织如下:

第一阶段(第1--4月):完成空间感知模块研发。里程碑:完成分割与区域退化识别联调,形成空间退化图生成流程。可交付件:模块代码、评测脚本、阶段实验报告。验收标准:完成区域级退化识别基线对比,关键指标优于全图识别基线。

第二阶段(第5--8月):完成层级多智能体协同规划系统。里程碑:打通全局调度、区域专家、质量反馈闭环。可交付件:智能体系统原型、消融实验结果、阶段论文初稿。验收标准:在混合退化任务上相较静态管线取得质量或效率改进。

第三阶段(第9--12月):完成轨迹蒸馏与轻量部署。里程碑:构建轨迹数据集并训练轻量路由网络,完成端侧测试。可交付件:蒸馏模型、部署报告、论文与专利材料。验收标准:在主要质量指标可接受损失范围内,显著降低推理时延与算力开销。

时间映射关系:研究内容1对应第一阶段,研究内容2对应第二阶段,研究内容3对应第三阶段;各阶段输出将逐步汇聚为最终系统。

\subsection{可行性分析}

\textbf{(1)理论可行性}

多智能体协同与大模型规划机制已在复杂任务求解中证明有效\cite{agenticir,restoreagent,mair,qagent};空间感知分析所依赖的SAM\cite{sam}与DepictQA\cite{depictqa}具备成熟的预训练基础;轨迹蒸馏作为知识压缩手段已在多任务学习中展现稳定性。上述理论基础与本项目“感知—规划—蒸馏”技术路线一致。

\textbf{(2)技术可行性}

在团队基础方面,已形成图像恢复与连续比例因子建模研究积累\cite{wu2023tgrs,wu2022cjig,wu2020tgrs};在平台条件方面,具备预训练模型接入、训练评测与可视化分析环境;在数据条件方面,具备构建合成退化与真实测试样本的能力。针对潜在瓶颈(算力开销、数据噪声、模型稳定性),已预设轻量路由替代、数据清洗与困难样本重训等缓释策略。

综上,本项目在理论与工程两个层面均具备实施条件,可支撑既定目标按计划推进。

