\section{思路方法}

\subsection{具体研究方法}

本项目的整体研究思路为“感知$\to$规划$\to$执行$\to$蒸馏”。首先构建空间感知模块实现区域级退化分析,然后设计层级智能体进行协同规划与执行,最后通过轨迹蒸馏实现轻量化部署。三个研究内容层层递进、相互依托。具体方法流程如图~\ref{fig:method-pipeline}~所示。

% === 图2 占位符 ===
% 【画图 AI Prompt(详细描述)】:
% 绘制一张学术论文风格的详细方法流程图,展示"多智能体协同的空间感知图像恢复"的
% 完整技术流程。白色背景,专业扁平化设计,整体布局为从上到下的三行结构,
% 行与行之间有清晰的数据流箭头连接。
%
% 【第一行:空间感知退化分析(浅蓝色背景带)】
%   从左到右排列:
%   (a) 输入:一张混合退化图像 y(城市街景,不同区域有不同退化:
%       天空区域有雾霾、建筑区域有噪声、道路区域有模糊、植被区域有雨线)。
%   (b) SAM分割步骤:图像被SAM模型分割为4-5个语义区域,
%       每个区域用不同颜色的半透明掩码覆盖(天空=蓝色掩码、建筑=橙色掩码、
%       道路=灰色掩码、植被=绿色掩码),区域边界用白色线条标出。
%       上方标注公式 "{R_i} = SAM(y)"。
%   (c) VLM逐区域分析步骤:展示为多个并行的分析通道,每个通道对应一个区域。
%       每个通道显示:区域裁剪图 → DepictQA模型图标 → 结构化输出。
%       用文字气泡展示分析结果示例:
%       "R1(天空): 雾霾=0.8, 噪声=0.1, 模糊=0.0"
%       "R2(建筑): 噪声=0.9, 雾霾=0.2, 模糊=0.1"
%       "R3(道路): 模糊=0.7, 噪声=0.3, 雾霾=0.0"
%       上方标注 "(d_i, s_i) = VLM(y_{R_i})"。
%   (d) 输出:空间退化图 M,以彩色热力图叠加在原图上展示,
%       每个区域标注其主要退化类型和严重程度数值。
%       旁边用数学符号标注 "M = {(R_i, d_i, s_i)}"。
%
% 【第二行:层级多智能体协同规划(浅橙色背景带)】
%   中央为双层架构示意:
%   (上层——全局调度器) 一个大的圆角矩形节点,内部显示:
%       左侧:接收空间退化图 M 作为输入(从第一行有粗箭头连下来)。
%       中部:退化先验排序规则的可视化——三个箭头从右到左排列,
%       分别标注"1.压缩退化(JPEG伪影)""2.成像退化(噪声/模糊)""3.场景退化(雾霾/雨线)",
%       表示逆序处理优先级。
%       右侧:输出各区域的处理优先级队列和策略分配。
%   (下层——区域专家组) 3个并排的小圆角矩形,分别标注"区域专家1""区域专家2""区域专家3"。
%       每个专家内部显示:
%       - 工具选择:从一个竖排的"工具库"(包含图标:Denoiser去噪、Deblurrer去模糊、
%         Dehazer去雾、Derainer去雨、SR超分辨率、JPEG去伪影)中选出2-3个工具,
%         用箭头连成工具链,例如 Expert1: Dehazer→Denoiser。
%       - 参数设定:每个选中工具旁有一个小滑块/旋钮,标注连续参数值
%         如 "σ=0.7""scale=2×"。
%       - 工具链右侧显示修复后的区域图像小缩略图。
%   (反馈回路) 每个区域专家的输出有箭头指向一个"质量评估 Q(·)"模块
%       (DepictQA图标),评估模块输出分数(如"Q=0.82"),
%       与阈值 τ 比较:若 Q < τ 则红色箭头反馈回区域专家(标注"调参重试"),
%       若 Q ≥ τ 则绿色箭头向右输出(标注"通过")。
%   (融合输出) 所有通过质量阈值的区域修复结果汇聚到一个"掩码加权融合 ⊕"节点,
%       输出最终恢复图像 x̂。公式标注 "x̂ = Σ R_i ⊙ x̂_{R_i}"。
%
% 【第三行:轨迹蒸馏与轻量部署(浅绿色背景带)】
%   左侧——数据收集:
%       从第二行有一条大的虚线箭头标注"离线运行 & 轨迹记录",
%       指向多组轨迹数据的示意:3-4个小卡片,每个卡片左侧是退化图像缩略图,
%       右侧是该图像对应的决策轨迹(区域划分 + 工具链 + 参数),
%       用文字简写如 "y→{R1:Dehaze(0.8)→Denoise(0.5), R2:Deblur(0.6)→SR(2×)}"。
%       标注 "轨迹数据集 T, |T|=M"。
%   中部——蒸馏训练:
%       一个"知识蒸馏"箭头(粗虚线箭头,标注"监督学习")从轨迹数据指向
%       一个轻量级网络结构示意图。该网络以退化图像为输入(左侧),
%       经过一个小型编码器(几层卷积块),输出四个分支头:
%       "分割头"(输出区域掩码)、"分类头"(输出退化类型)、
%       "路由头"(输出工具选择)、"参数头"(输出连续参数)。
%       网络标注为 "Φ_ω"。损失函数标注在网络下方:
%       "L = L_seg + λ1·L_cls + λ2·L_route + λ3·L_param"。
%   右侧——部署推理:
%       轻量路由网络 Φ_ω 直接连接工具库执行恢复,输出最终图像。
%       旁边标注关键优势:"无需LLM/VLM"(旁有一个被划掉的大模型图标)、
%       "实时推理"(旁有一个时钟/闪电图标)。
%
% 【全局标注】:
%   三行左侧分别标注研究内容编号 "(1)""(2)""(3)"。
%   行与行之间的连接箭头标注传递的数据名称。
%   整体配色:蓝色系(感知)、橙色系(规划)、绿色系(部署),
%   与图1的三模块颜色保持一致。
%
\vspace{6pt}
\begin{figure}[H]
\centering
\includegraphics[width=\linewidth]{imgs/Gemini_Generated_Image_qjhpc4qjhpc4qjhp.png}\\[2pt]
\caption{具体研究方法流程}
\label{fig:method-pipeline}
\end{figure}
\vspace{6pt}

\textbf{(1)空间感知退化分析}

给定退化图像$\mathbf{y} \in \mathbb{R}^{H \times W \times 3}$,本模块旨在生成空间退化图$\mathcal{M}$,为后续智能体规划提供结构化输入。

\textbf{步骤一:语义区域分割。}利用零样本分割模型SAM\cite{sam}将图像划分为$N$个语义一致的区域:
\begin{equation}
\{R_i\}_{i=1}^{N} = \mathrm{SAM}(\mathbf{y}), \quad \bigcup_{i=1}^{N} R_i = \Omega, \quad R_i \cap R_j = \emptyset \ (i \neq j)
\end{equation}
其中$\Omega$为图像像素域。SAM基于超过10亿掩码的预训练,具备良好的零样本泛化能力。然而,直接应用于退化图像时,模糊边界、噪声纹理和对比度损失会导致分割质量下降。为此,本研究引入置信度门控机制:
\begin{equation}
R_i^{\text{final}} = \begin{cases} R_i, & \text{if } \mathcal{C}(R_i) > \tau_{\text{seg}} \\ R_i^{\text{robust}}, & \text{otherwise} \end{cases}
\end{equation}
其中$\mathcal{C}(R_i)$为SAM对区域$R_i$的分割置信度,$\tau_{\text{seg}}$为预设阈值。当置信度低于阈值时,启用RobustSAM\cite{robustsam}重新分割,其在噪声、模糊、低对比度等退化条件下保持稳健性能。

\textbf{步骤二:区域级退化分析。}对每个分割区域$R_i$,提取其裁剪图像$\mathbf{y}_{R_i} = \mathbf{y} \odot \mathbf{M}_i$,其中$\mathbf{M}_i \in \{0,1\}^{H \times W}$为区域掩码。将$\mathbf{y}_{R_i}$输入经Q-Instruct数据集微调的视觉语言模型$V_{\theta}$,采用链式思维推理逐类判断退化:
\begin{equation}
\mathbf{d}_i^{(c)}, s_i^{(c)} = V_{\theta}\left(\text{Prompt}_c, \mathbf{y}_{R_i}\right), \quad c = 1, \ldots, C
\end{equation}
其中$\text{Prompt}_c$为针对第$c$类退化的查询提示。为提升分析精度,引入空间上下文聚合:
\begin{equation}
\mathbf{h}_i = \text{ContextAgg}\left(\{V_{\theta}(\mathbf{y}_{R_j})\}_{j \in \mathcal{N}(i)}\right)
\end{equation}
其中$\mathcal{N}(i)$表示区域$R_i$的空间邻域集合,通过图注意力机制聚合邻域退化信息,修正孤立区域的误判。

\textbf{步骤三:空间退化图构建。}汇总所有区域分析结果,构建结构化空间退化图:
\begin{equation}
\mathcal{M} = \left\{(R_i, \mathbf{d}_i, \mathbf{s}_i, \mathbf{h}_i)\right\}_{i=1}^{N}
\end{equation}
该图显式编码了退化的空间分布、类型判别与严重程度估计,可直接作为后续决策模块的状态描述。

\textbf{验证方案:}在SIDD去噪、GoPro去模糊、NH-HAZE去雾等公开基准上,采用分类准确率(Acc)、平均精度(mAP)和F1分数评估退化识别性能。对比方案包括:(1)全图级DepictQA分析基线;(2)SAM分割 + 独立区域分析(无上下文聚合);(3)本研究完整方法。通过消融实验量化各组件(RobustSAM回退、上下文聚合)对最终恢复质量(PSNR/SSIM)的贡献。

\textbf{(2)层级多智能体协同规划}

基于空间退化图$\mathcal{M}$,本研究设计全局调度器$G_{\phi}$与区域专家$\{E_i\}_{i=1}^{N}$的协同架构。全局调度器负责跨区域策略协调,区域专家负责区域内恢复决策。

\textbf{步骤一:全局调度与优先级分配。}全局调度器依据退化先验(场景退化$\to$成像退化$\to$压缩退化的逆序处理原则\cite{mair}),为各区域分配处理优先级。定义退化优先级函数:
\begin{equation}
\mathcal{P}(R_i) = \sum_{c=1}^{C} w_c \cdot d_i^{(c)} \cdot s_i^{(c)}
\end{equation}
其中$w_c$为第$c$类退化的先验权重(场景退化$w_c > 1$,成像退化$w_c = 1$,压缩退化$0 < w_c < 1$)。全局调度器按$\mathcal{P}(R_i)$降序排列区域处理顺序,并生成全局策略向量:
\begin{equation}
\mathbf{g} = G_{\phi}(\mathcal{M}) = \text{Softmax}\left(\text{MLP}\left(\{\mathbf{h}_i\}_{i=1}^{N}\right)\right)
\end{equation}
其中$\mathbf{g} \in \mathbb{R}^{N}$编码各区域的处理优先级权重。

\textbf{步骤二:区域专家工具链选择与参数优化。}对于区域$R_i$,区域专家$E_i$从预构建的工具库$\mathcal{F} = \{f_1, f_2, \ldots, f_L\}$中选择恢复工具序列$\pi_i = (f_{\sigma_1}, \ldots, f_{\sigma_{K_i}})$,并为每个工具设定连续参数$\boldsymbol{\theta}_i = (\theta_{\sigma_1}, \ldots, \theta_{\sigma_{K_i}})$。区域修复结果通过函数复合计算:
\begin{equation}
\hat{\mathbf{x}}_{R_i} = (f_{\sigma_{K_i}} \circ \cdots \circ f_{\sigma_1})(\mathbf{y}_{R_i}; \boldsymbol{\theta}_i)
\end{equation}

工具链选择采用策略梯度方法优化。定义策略网络$\pi_{\psi}(\cdot | \mathbf{h}_i, \mathbf{g})$输出工具选择概率分布,通过最大化期望恢复质量进行训练:
\begin{equation}
\max_{\psi} \mathbb{E}_{\pi_i \sim \pi_{\psi}}\left[Q\left((f_{\sigma_{K_i}} \circ \cdots \circ f_{\sigma_1})(\mathbf{y}_{R_i})\right)\right] - \beta \cdot \text{Length}(\pi_i)
\end{equation}
其中$Q(\cdot)$为质量评估函数,$\beta$为工具链长度惩罚系数,鼓励简洁高效的处理流程。

\textbf{步骤三:连续参数自适应优化。}对于选定的工具链$\pi_i$,区域专家通过迭代优化确定最优参数:
\begin{equation}
\boldsymbol{\theta}_i^{*} = \arg\max_{\boldsymbol{\theta}_i} Q\left(\hat{\mathbf{x}}_{R_i}(\boldsymbol{\theta}_i)\right), \quad \text{s.t.} \quad Q\left(\hat{\mathbf{x}}_{R_i}(\boldsymbol{\theta}_i)\right) \geq \tau
\end{equation}
其中$\tau$为质量阈值。采用贝叶斯优化或梯度上升法求解,迭代更新直至收敛或达到最大迭代次数$T_{\max}$。

\textbf{步骤四:全局融合与一致性约束。}全局调度器通过区域掩码加权融合各区域结果:
\begin{equation}
\hat{\mathbf{x}} = \sum_{i=1}^{N} \mathbf{M}_i \odot \hat{\mathbf{x}}_{R_i} + \lambda_{\text{smooth}} \cdot \nabla^2 \hat{\mathbf{x}}
\end{equation}
其中$\lambda_{\text{smooth}}$为平滑正则系数,通过拉普拉斯算子约束边界一致性。全局质量评估通过一致性损失实现跨区域协调:
\begin{equation}
\mathcal{L}_{\text{consist}} = \sum_{(i,j) \in \mathcal{E}} \left\| \hat{\mathbf{x}}_{R_i}^{\text{border}} - \hat{\mathbf{x}}_{R_j}^{\text{border}} \right\|_2^2
\end{equation}
其中$\mathcal{E}$为相邻区域边界的像素集合。

\textbf{验证方案:}在混合退化合成数据集(通过SIDD、GoPro、NH-HAZE等组合生成)与真实测试样本上,与AirNet、PromptIR等All-in-One静态方法以及AgenticIR、MAIR等智能体方法进行全面对比。评估指标包括:(1)恢复质量:PSNR、SSIM、LPIPS;(2)推理效率:FPS、平均工具调用次数;(3)决策质量:工具链选择准确率、参数优化收敛速度。通过消融实验验证层级架构各组件(全局调度、区域专家、质量反馈、一致性约束)的贡献度。

\textbf{(3)基于轨迹蒸馏的轻量路由网络}

\textbf{步骤一:决策轨迹数据集构建。}利用上述多智能体系统在大规模合成退化数据集$\mathcal{D}_{\text{syn}} = \{(\mathbf{y}^{(j)}, \mathbf{x}^{(j)})\}_{j=1}^{M}$上离线运行,收集决策轨迹数据集:
\begin{equation}
\mathcal{T} = \left\{\left(\mathbf{y}^{(j)}, \mathcal{M}^{(j)}, \left\{\pi_i^{*,(j)}, \boldsymbol{\theta}_i^{*,(j)}, \mathbf{M}_i^{(j)}\right\}_{i=1}^{N_j}\right)\right\}_{j=1}^{M}
\end{equation}
其中每条轨迹记录输入图像$\mathbf{y}^{(j)}$、空间退化图$\mathcal{M}^{(j)}$、各区域最优工具序列$\pi_i^{*}$、最优参数$\boldsymbol{\theta}_i^{*}$以及区域掩码$\mathbf{M}_i$。

为提升轨迹质量,引入结果过滤机制:仅保留最终恢复质量$Q(\hat{\mathbf{x}}^{(j)}) \geq \tau_{\text{high}}$的轨迹参与蒸馏。同时,采用轨迹多样性采样确保退化类型、区域数量、工具链长度的分布均衡。

\textbf{步骤二:轻量路由网络架构设计。}设计轻量级路由网络$\Phi_{\omega}$,采用编码器—多任务解码器架构:

编码器采用轻量化骨干网络(如MobileNet-V3或EfficientNet-B0)提取多尺度特征金字塔$\{\mathbf{F}_l\}_{l=1}^{L}$,其中$\mathbf{F}_l \in \mathbb{R}^{H/2^l \times W/2^l \times C_l}$。

解码器分为四个并行分支头:

(a) \textbf{分割头}$\Phi_{\omega}^{\text{seg}}$:输出区域掩码概率图$\hat{\mathbf{P}}^{\text{mask}} \in \mathbb{R}^{H \times W \times N_{\max}}$,通过Softmax获取区域归属概率:
\begin{equation}
\hat{\mathbf{M}}_i = \mathbb{1}\left[\arg\max_k \hat{P}^{\text{mask}}(u,v,k) = i\right]
\end{equation}

(b) \textbf{分类头}$\Phi_{\omega}^{\text{cls}}$:对每个区域输出退化类型概率分布:
\begin{equation}
\hat{\mathbf{d}}_i = \text{Sigmoid}\left(\text{GAP}\left(\mathbf{F}_{\text{roi}}^{(i)}\right)\right) \in [0,1]^{C}
\end{equation}
其中$\text{GAP}$为全局平均池化,$\mathbf{F}_{\text{roi}}^{(i)}$为区域$R_i$对应的RoI特征。

(c) \textbf{路由头}$\Phi_{\omega}^{\text{route}}$:输出工具选择概率矩阵:
\begin{equation}
\hat{\mathbf{P}}^{\text{route}}_i = \text{Softmax}\left(\text{MLP}\left(\mathbf{F}_{\text{roi}}^{(i)} \oplus \hat{\mathbf{d}}_i\right)\right) \in \mathbb{R}^{L}
\end{equation}
其中$\oplus$表示特征拼接。

(d) \textbf{参数头}$\Phi_{\omega}^{\text{param}}$:对每个选中工具输出连续参数值:
\begin{equation}
\hat{\boldsymbol{\theta}}_i = \text{Sigmoid}\left(\text{MLP}\left(\mathbf{F}_{\text{roi}}^{(i)}\right)\right) \cdot \boldsymbol{\theta}_{\max}
\end{equation}
其中$\boldsymbol{\theta}_{\max}$为各参数的最大允许值。

\textbf{步骤三:多任务联合蒸馏训练。}基于轨迹数据集进行监督学习,联合优化四个任务:
\begin{equation}
\mathcal{L} = \mathcal{L}_{\text{seg}} + \lambda_1 \mathcal{L}_{\text{cls}} + \lambda_2 \mathcal{L}_{\text{route}} + \lambda_3 \mathcal{L}_{\text{param}} + \lambda_4 \mathcal{L}_{\text{kd}}
\end{equation}

各损失项定义如下:
\begin{align}
\mathcal{L}_{\text{seg}} &= \frac{1}{M} \sum_{j=1}^{M} \sum_{i=1}^{N_j} \text{Dice}\left(\hat{\mathbf{M}}_i^{(j)}, \mathbf{M}_i^{*,(j)}\right) \\
\mathcal{L}_{\text{cls}} &= \frac{1}{M} \sum_{j=1}^{M} \sum_{i=1}^{N_j} \text{BCE}\left(\hat{\mathbf{d}}_i^{(j)}, \mathbf{d}_i^{*,(j)}\right) \\
\mathcal{L}_{\text{route}} &= \frac{1}{M} \sum_{j=1}^{M} \sum_{i=1}^{N_j} \sum_{k=1}^{K_i} \text{CE}\left(\hat{\mathbf{P}}^{\text{route}}_{i,k}, f_{\sigma_k}^{*,(j)}\right) \\
\mathcal{L}_{\text{param}} &= \frac{1}{M} \sum_{j=1}^{M} \sum_{i=1}^{N_j} \sum_{k=1}^{K_i} \left\|\hat{\boldsymbol{\theta}}_{i,k}^{(j)} - \boldsymbol{\theta}_{i,k}^{*,(j)}\right\|_2^2
\end{align}
其中$\lambda_1, \lambda_2, \lambda_3, \lambda_4$为任务平衡系数。知识蒸馏损失$\mathcal{L}_{\text{kd}}$用于保留多智能体的“软”决策信息:
\begin{equation}
\mathcal{L}_{\text{kd}} = \text{KL}\left(\text{Softmax}(\hat{\mathbf{z}} / T) \| \text{Softmax}(\mathbf{z}^{*} / T)\right)
\end{equation}
其中$\hat{\mathbf{z}}$和$\mathbf{z}^{*}$分别为路由网络和多智能体的logits输出,$T$为温度系数。

\textbf{步骤四:困难样本再训练与部署优化。}针对蒸馏偏差问题,实施两阶段训练策略:

第一阶段:在全量轨迹数据上训练至收敛。

第二阶段:识别困难样本集$\mathcal{T}_{\text{hard}} = \{(\mathbf{y}^{(j)}, \cdot) \in \mathcal{T} : Q(\hat{\mathbf{x}}^{(j)}) < Q(\mathbf{x}_{\text{agent}}^{(j)}) - \delta\}$,即路由网络与多智能体性能差距超过阈值$\delta$的样本。在该子集上进行困难样本聚焦训练,损失函数加权:
\begin{equation}
\mathcal{L}_{\text{hard}} = \sum_{(\mathbf{y}^{(j)}, \cdot) \in \mathcal{T}_{\text{hard}}} w_j \cdot \mathcal{L}^{(j)}, \quad w_j \propto Q(\mathbf{x}_{\text{agent}}^{(j)}) - Q(\hat{\mathbf{x}}^{(j)})
\end{equation}

部署阶段,轻量路由网络直接根据输入图像预测区域划分、退化类型、工具调度与参数配置,无需LLM/VLM在线推理,实现端到端实时恢复:
\begin{equation}
\left\{\hat{\mathbf{M}}_i, \hat{\mathbf{d}}_i, \hat{\pi}_i, \hat{\boldsymbol{\theta}}_i\right\}_{i=1}^{\hat{N}} = \Phi_{\omega}(\mathbf{y})
\end{equation}

\textbf{验证方案:}在测试集上对比蒸馏后的轻量路由网络与原多智能体系统的性能差距,要求PSNR损失$\leq 5\%$、SSIM损失$\leq 3\%$。测试部署时延(ms/图像)、模型参数量(MB)、计算量(FLOPs),验证在Jetson Nano或同等边缘设备上的实时推理可行性(目标:$\geq 10$ FPS @ 720p)。通过消融实验验证困难样本再训练对性能提升的贡献,分析不同退化类型、区域数量对蒸馏效果的影响。


\subsection{研究计划}

本项目的计划研究时间为2年,具体研究计划如下:

\textbf{第一阶段(第1–6月)}

\begin{itemize}
\item 完成空间感知退化分析的基础架构,实现SAM分割与VLM退化识别的联调;
\item 引入上下文聚合与RobustSAM回退机制,建立区域级退化识别评测体系;
\item 在SIDD、GoPro、NH-HAZE等基准上验证退化类型识别准确率;
\end{itemize}

\textbf{第二阶段(第7–12月)}

\begin{itemize}
\item 完成层级多智能体协同架构搭建,实现全局调度器与区域专家的基础功能;
\item 引入全局一致性约束与早停机制,与AirNet、PromptIR等方法完成系统性对比;
\item 构建混合退化合成数据集,完成发明专利交底书撰写;
\item 向本领域国内外权威学术期刊或重要学术会议投稿论文1–2篇;
\end{itemize}

\textbf{第三阶段(第13–18月)}

\begin{itemize}
\item 构建规模不少于10000条的多智能体决策轨迹数据集;
\item 完成轻量路由网络的多任务联合蒸馏,实现端到端推理基础版本;
\item 针对困难样本进行聚焦训练,缩小与多智能体系统的性能差距;
\end{itemize}

\textbf{第四阶段(第19–24月)}

\begin{itemize}
\item 完成轻量路由网络的部署优化,在边缘设备(Jetson Nano或同等级平台)完成测试;
\item 实现系统集成,形成完整的感知—规划—执行—蒸馏闭环;
\item 向本领域国内外权威学术期刊或重要学术会议投稿论文1–2篇;
\item 撰写项目结题报告,汇总全部研究成果。
\end{itemize}

\subsection{可行性分析}

\textbf{(1)理论可行性}

多智能体协同与大模型规划机制已在复杂任务求解中证明有效\cite{agenticir,restoreagent,mair,qagent};空间感知分析所依赖的SAM\cite{sam}与DepictQA\cite{depictqa}具备成熟的预训练基础;轨迹蒸馏作为知识压缩手段已在多任务学习中展现稳定性。上述理论基础与本项目“感知—规划—蒸馏”技术路线一致。

\textbf{(2)技术可行性}

在团队基础方面,已形成图像恢复与连续比例因子建模研究积累\cite{wu2023tgrs,wu2022cjig,wu2020tgrs};在平台条件方面,具备预训练模型接入、训练评测与可视化分析环境;在数据条件方面,具备构建合成退化与真实测试样本的能力。针对潜在瓶颈(算力开销、数据噪声、模型稳定性),已预设轻量路由替代、数据清洗与困难样本重训等缓释策略。

综上,本项目在理论与工程两个层面均具备实施条件,可支撑既定目标按计划推进。
