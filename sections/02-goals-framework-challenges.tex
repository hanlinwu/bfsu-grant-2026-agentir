\section{研究目标、总体框架及重点难点}

本节围绕“目标—内容—难点”建立一一映射关系:目标定义“做成什么”,研究内容定义“怎么做”,重点难点定义“最难在哪里以及如何突破”,三者共同支撑项目可执行性。

\subsection{研究目标}

面向复杂退化场景下图像恢复的精细化与高效化需求,本项目拟实现以下可评价目标:

1)\textbf{目标1(感知)}:构建空间感知退化分析机制,实现区域级退化识别与量化;在公开复杂退化基准上,相较全图级识别基线,区域级退化识别准确率显著提升。对应输出物:空间退化图构建算法、区域级标注与评测脚本、阶段论文成果。

2)\textbf{目标2(决策)}:构建层级多智能体协同机制,实现区域级恢复规划与连续参数自适应优化;在混合退化恢复任务中,相较静态恢复管线获得更高恢复质量与更优效率平衡。对应输出物:层级智能体恢复系统原型、工具链调度策略、发明专利/论文成果。

3)\textbf{目标3(部署)}:构建轨迹蒸馏驱动的轻量路由模型,实现高质量恢复策略的快速推理与边缘部署;在保持主要质量指标可接受损失范围内显著降低部署时延与算力需求。对应输出物:轻量路由网络模型、部署测试报告、可复现实验配置。

\subsection{总体框架}

本项目总体框架如图1所示,由”空间感知模块—层级规划模块—轻量部署模块”三大核心模块组成。空间感知模块负责将退化图像分割为语义区域并生成空间退化图;层级规划模块基于全局调度器与区域专家的协同架构,对各区域独立规划恢复策略并迭代优化;轻量部署模块通过轨迹蒸馏将多智能体的复杂决策知识压缩为轻量路由网络。三个模块层层递进,形成”感知—决策—执行—蒸馏”的闭环流程。

其中,系统输入为退化图像$\mathbf{y}$;模块一输出空间退化图$\mathcal{M}$;模块二输出区域修复结果及全图恢复结果$\hat{\mathbf{x}}$;模块三输出轻量推理模型$\Phi_\omega$及对应调度参数。三模块在训练阶段通过轨迹数据闭环耦合,在部署阶段形成“轻量网络 + 工具库”的高效执行路径。

% === 图1 占位符 ===
% 【画图 AI Prompt(详细描述)】:
% 绘制一张学术论文风格的水平流程图,展示”多智能体协同的空间感知图像恢复系统”的总体框架。白色背景,专业扁平化设计,无3D效果。
%
% 整体布局:从左到右的水平管线流程,分为输入、三个核心模块、输出五个部分。
%
% 【输入端(最左侧)】:一张退化图像,图像中不同空间区域呈现不同的退化类型——
%   左上区域有明显的高斯噪声颗粒,中部区域有运动模糊拖影,下方区域被雾霾笼罩呈灰白色,
%   右侧有雨线条纹。图像下方标注”退化图像 y”。
%
% 【模块一(浅蓝色圆角矩形,标题栏写”空间感知模块”)】:
%   内部自上而下排列三个步骤,用箭头串联:
%   (a) “SAM分割”:图标为一个分割网络示意,旁边显示退化图像被分割为4-5个彩色区域
%       (天空=浅蓝、建筑=橙色、道路=灰色、植被=绿色),区域边界用彩色虚线标出。
%   (b) “VLM退化分析”:图标为DepictQA模型,旁边用气泡框展示链式推理过程,
%       例如”区域1:噪声=强,模糊=无,雾霾=中等”。
%   (c) 输出:一张”空间退化图 M”,用热力图/彩色编码叠加在原图上,
%       不同颜色代表不同退化类型(红色=噪声,蓝色=模糊,灰色=雾霾,黄色=雨线),
%       颜色深浅表示严重程度。旁边标注数学符号 M = {(R_i, d_i, s_i)}。
%
% 【模块二(浅橙色圆角矩形,标题栏写”层级规划模块”)】:
%   内部为两层架构:
%   (上层) “全局调度器”:一个大的控制节点,图标为大脑/指挥中心,标注”LLM”。
%       旁边显示退化先验排序规则:”压缩退化→成像退化→场景退化”(逆序处理箭头)。
%       从空间退化图 M 有箭头输入到全局调度器。
%   (下层) 3-4个”区域专家”:每个为一个小智能体图标,分别标注”Expert 1””Expert 2””Expert 3”。
%       全局调度器与各区域专家之间有双向箭头(指令下发↓,结果上报↑)。
%       每个区域专家连接一个”工具库”(竖排图标列表:去噪器Denoiser、去模糊器Deblurrer、
%       去雾器Dehazer、去雨器Derainer、超分辨率SR),每个工具旁有参数滑块(θ)。
%   (反馈环) 从修复结果有箭头指向”质量评估 Q(·)”模块,再反馈回区域专家,
%       形成闭环,标注”迭代优化”。
%   (输出) 各区域修复结果通过”掩码融合 ⊕”节点合成为完整修复图像。
%
% 【模块三(浅绿色圆角矩形,标题栏写”轻量部署模块”)】:
%   (上方) 从模块二有一条虚线箭头标注”离线轨迹收集”,指向多组”(输入图像→决策轨迹)”
%       数据对的示意图标。
%   (中部) “轨迹蒸馏”箭头指向一个轻量级CNN网络结构示意图,标注”路由网络 Φ_ω”。
%   (下方) 一个被红叉划掉的LLM图标,旁边文字”无需大模型”,表示部署时不依赖LLM/VLM。
%       路由网络直接输出:区域掩码 + 退化类型 + 工具路由 + 参数。
%
% 【输出端(最右侧)】:一张清晰的高质量恢复图像,各区域退化均已消除,
%   色彩鲜明,细节清晰。图像下方标注”恢复图像 x̂”。
%
% 【全局连接】:三个模块之间用粗实线箭头顺序连接。模块一→模块二标注”空间退化图”,
%   模块二→输出标注”恢复图像”。模块二到模块三用虚线箭头标注”轨迹数据”。
%   模块三→输出有一条虚线标注”实时推理路径(部署阶段)”。
%
\vspace{6pt}
\begin{center}
\includegraphics[width=\linewidth]{imgs/Gemini_Generated_Image_lk4hu6lk4hu6lk4h.png}\\[2pt]
{\small 图1\quad 本项目总体研究框架}
\end{center}
\vspace{6pt}

\subsection{研究内容}

面向上述目标,本项目拟开展以下三个方面的研究。

\textbf{1)空间感知退化分析}

研究问题:现有方法主要基于全图级退化判断,难以刻画复杂场景的空间异质退化。关键方法:引入零样本分割模型SAM\cite{sam}进行语义区域划分,并结合DepictQA\cite{depictqa}进行区域级退化类型与严重程度估计,构建空间退化图。预期效果:为后续规划提供可解释、可量化的结构化输入。与研究内容2的依赖关系:本模块输出$\mathcal{M}$作为层级智能体的决策先验。创新嵌入点:以区域粒度统一表征“退化类型-程度-位置”。

\textbf{2)层级多智能体协同规划}

研究问题:单一静态恢复流程难以兼顾复杂退化下的质量与效率。关键方法:构建“全局调度器 + 区域专家”的两层智能体架构,全局层依据退化先验\cite{mair}分配处理优先级,区域层执行工具链选择与连续参数优化,并通过质量反馈迭代更新。预期效果:实现区域差异化修复与全图一致性兼顾。与研究内容1/3的依赖关系:依赖$\mathcal{M}$完成规划,并为轨迹蒸馏提供高质量决策样本。创新嵌入点:实现工具调用与参数调节的一体化决策。

\textbf{3)基于轨迹蒸馏的轻量路由网络}

研究问题:多智能体高质量推理成本高、部署门槛高。关键方法:离线收集“输入图像$\to$最优路径与参数”轨迹数据,训练轻量路由网络直接预测分割、退化类型与工具调度方案。预期效果:在保持恢复质量可接受范围内大幅降低部署时延与算力开销。与研究内容2的依赖关系:蒸馏对象来自多智能体轨迹;反向促进规划策略标准化。创新嵌入点:将复杂多步推理知识压缩为可部署决策网络。

\subsection{重点难点}

本项目的重点难点包括:

1)\textbf{难点1(对应目标1)}:如何在零样本条件下准确分割退化区域并量化退化程度。拟解决路径:通过“语义分割 + 区域级退化推理 + 一致性约束”联合优化空间退化图质量。

2)\textbf{难点2(对应目标2)}:如何在空间维度上协调多智能体决策,避免区域边界不一致与策略冲突。拟解决路径:通过全局调度优先级约束与区域专家反馈闭环实现跨区域协同。

3)\textbf{难点3(对应目标3)}:如何将复杂多步推理知识有效压缩到轻量网络中并保持性能稳定。拟解决路径:采用多任务联合蒸馏、轨迹筛选与部署约束联合训练。

以上难点分别对应“感知—决策—部署”三类核心目标,并在第3节中给出具体技术实现路径。

