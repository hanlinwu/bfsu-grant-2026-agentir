\section{研究目标、总体框架及重点难点}

本节围绕"目标—内容—难点"建立一一映射关系:目标定义"做成什么",研究内容定义"怎么做",重点难点定义"最难在哪里以及如何突破",三者共同支撑项目可执行性。

\subsection{研究目标}

面向复杂退化场景下图像恢复的精细化与高效化需求,本项目拟实现以下可评价目标:

1)\textbf{目标1(感知)}:构建空间感知退化分析机制,实现区域级退化识别与量化。在公开复杂退化基准上,相较全图级识别基线,区域级退化识别准确率显著提升。对应输出物:空间退化图构建算法、区域级标注与评测脚本、阶段论文成果。

2)\textbf{目标2(决策)}:构建层级多智能体协同机制,实现区域级恢复规划与连续参数自适应优化。在混合退化恢复任务中,相较静态恢复管线获得更高恢复质量与更优效率平衡。对应输出物:层级智能体恢复系统原型、工具链调度策略、发明专利/论文成果。

3)\textbf{目标3(部署)}:构建轨迹蒸馏驱动的轻量路由模型,实现高质量恢复策略的快速推理与边缘部署。在保持主要质量指标可接受损失范围内显著降低部署时延与算力需求。对应输出物:轻量路由网络模型、部署测试报告、可复现实验配置。

\subsection{总体框架}

本项目总体框架如图~\ref{fig:overall-framework}所示,由"空间感知模块—层级规划模块—轻量部署模块"三大核心模块组成。空间感知模块负责将退化图像分割为语义区域并生成空间退化图;层级规划模块基于全局调度器与区域专家的协同架构,对各区域独立规划恢复策略并迭代优化;轻量部署模块通过轨迹蒸馏将多智能体的复杂决策知识压缩为轻量路由网络。三个模块层层递进,形成"感知—决策—执行—蒸馏"的闭环流程。

其中,系统输入为退化图像$\mathbf{y}$;模块一输出空间退化图$\mathcal{M}$;模块二输出区域修复结果及全图恢复结果$\hat{\mathbf{x}}$;模块三输出轻量推理模型$\Phi_\omega$及对应调度参数。三模块在训练阶段通过轨迹数据闭环耦合,在部署阶段形成"轻量网络 + 工具库"的高效执行路径。

% === 图1 占位符 ===
% 【画图 AI Prompt(详细描述)】:
% 绘制一张学术论文风格的水平流程图,展示"多智能体协同的空间感知图像恢复系统"的总体框架。白色背景,专业扁平化设计,无3D效果。
%
% 整体布局:从左到右的水平管线流程,分为输入、三个核心模块、输出五个部分。
%
% 【输入端(最左侧)】:一张退化图像,图像中不同空间区域呈现不同的退化类型——
%   左上区域有明显的高斯噪声颗粒,中部区域有运动模糊拖影,下方区域被雾霾笼罩呈灰白色,
%   右侧有雨线条纹。图像下方标注"退化图像 y"。
%
% 【模块一(浅蓝色圆角矩形,标题栏写"空间感知模块")】:
%   内部自上而下排列三个步骤,用箭头串联:
%   (a) "SAM分割":图标为一个分割网络示意,旁边显示退化图像被分割为4-5个彩色区域
%       (天空=浅蓝、建筑=橙色、道路=灰色、植被=绿色),区域边界用彩色虚线标出。
%   (b) "VLM退化分析":图标为DepictQA模型,旁边用气泡框展示链式推理过程,
%       例如"区域1:噪声=强,模糊=无,雾霾=中等"。
%   (c) 输出:一张"空间退化图 M",用热力图/彩色编码叠加在原图上,
%       不同颜色代表不同退化类型(红色=噪声,蓝色=模糊,灰色=雾霾,黄色=雨线),
%       颜色深浅表示严重程度。旁边标注数学符号 M = {(R_i, d_i, s_i)}。
%
% 【模块二(浅橙色圆角矩形,标题栏写"层级规划模块")】:
%   内部为两层架构:
%   (上层) "全局调度器":一个大的控制节点,图标为大脑/指挥中心,标注"LLM"。
%       旁边显示退化先验排序规则:"压缩退化→成像退化→场景退化"(逆序处理箭头)。
%       从空间退化图 M 有箭头输入到全局调度器。
%   (下层) 3-4个"区域专家":每个为一个小智能体图标,分别标注"Expert 1""Expert 2""Expert 3"。
%       全局调度器与各区域专家之间有双向箭头(指令下发↓,结果上报↑)。
%       每个区域专家连接一个"工具库"(竖排图标列表:去噪器Denoiser、去模糊器Deblurrer、
%       去雾器Dehazer、去雨器Derainer、超分辨率SR),每个工具旁有参数滑块(θ)。
%   (反馈环) 从修复结果有箭头指向"质量评估 Q(·)"模块,再反馈回区域专家,
%       形成闭环,标注"迭代优化"。
%   (输出) 各区域修复结果通过"掩码融合 ⊕"节点合成为完整修复图像。
%
% 【模块三(浅绿色圆角矩形,标题栏写"轻量部署模块")】:
%   (上方) 从模块二有一条虚线箭头标注"离线轨迹收集",指向多组"(输入图像→决策轨迹)"
%       数据对的示意图标。
%   (中部) "轨迹蒸馏"箭头指向一个轻量级CNN网络结构示意图,标注"路由网络 Φ_ω"。
%   (下方) 一个被红叉划掉的LLM图标,旁边文字"无需大模型",表示部署时不依赖LLM/VLM。
%       路由网络直接输出:区域掩码 + 退化类型 + 工具路由 + 参数。
%
% 【输出端(最右侧)】:一张清晰的高质量恢复图像,各区域退化均已消除,
%   色彩鲜明,细节清晰。图像下方标注"恢复图像 x̂"。
%
% 【全局连接】:三个模块之间用粗实线箭头顺序连接。模块一→模块二标注"空间退化图",
%   模块二→输出标注"恢复图像"。模块二到模块三用虚线箭头标注"轨迹数据"。
%   模块三→输出有一条虚线标注"实时推理路径(部署阶段)"。
%
\vspace{6pt}
\begin{figure}[H]
\centering
\includegraphics[width=\linewidth]{imgs/Gemini_Generated_Image_lk4hu6lk4hu6lk4h.png}\\[2pt]
\caption{本项目总体研究框架}
\label{fig:overall-framework}
\end{figure}
\vspace{6pt}

\subsection{研究内容}

面向上述目标,本项目拟开展以下三个方面的研究。

\textbf{(一)空间感知退化分析}

现有方法主要基于全图级退化判断,难以刻画复杂场景的空间异质退化。针对这一问题,本研究将构建区域级退化感知机制,实现"退化类型—严重程度—空间位置"的联合建模。

本研究首先利用零样本分割模型SAM\cite{sam}将退化图像划分为若干语义一致的区域。SAM基于超过10亿掩码的预训练,具备良好的零样本泛化能力,但直接应用于退化图像时存在分割质量下降问题。为此,本研究将引入RobustSAM\cite{robustsam}作为备选方案,其在噪声、模糊、低对比度等退化条件下保持稳健分割性能。

对每个分割区域,将其裁剪图像输入经Q-Instruct数据集微调的视觉语言模型,采用链式思维推理逐类判断退化是否存在及其程度,输出退化类型判别与对应严重程度估计。为提升分析精度,引入空间上下文聚合机制,通过图注意力机制聚合邻域退化信息,修正孤立区域的误判。汇总所有区域分析结果,构建空间退化图,为后续智能体规划提供结构化输入。

\textbf{(二)层级多智能体协同规划}

单一静态恢复流程难以兼顾复杂退化下的质量与效率。针对这一问题,本研究将构建"全局调度器 + 区域专家"的两层智能体架构,实现工具链选择与连续参数优化的一体化决策。

全局调度器依据退化先验(场景退化$\to$成像退化$\to$压缩退化的逆序处理原则\cite{mair}),为各区域分配处理优先级并制定整体修复策略。该先验基于物理成像过程的逆向逻辑:压缩退化(JPEG伪影)发生在图像编码阶段,应在最后处理;成像退化(噪声、模糊)源于采集设备;场景退化(雾霾、雨线)来自环境因素,应优先处理以还原场景辐射。

每个区域由专属的区域专家负责处理。区域专家从预构建的工具库中选择合适的恢复工具序列,并为每个工具设定连续参数(如去噪强度、超分辨率尺度)。工具库包含去噪器、去模糊器、去雾器、去雨器、超分辨率模型、JPEG伪影消除器等基础恢复模块。区域专家通过迭代方式优化工具参数,直至修复结果满足质量要求。

全局调度器通过区域掩码加权融合各区域修复结果,合成最终恢复图像,并对全图进行整体质量评估与一致性检查,确保跨区域边界处的平滑过渡。

\textbf{(三)基于轨迹蒸馏的轻量路由网络}

多智能体高质量推理依赖LLM/VLM在线调用,导致时延和算力成本偏高,不利于边缘部署与实时应用。针对这一问题,本研究将构建轨迹蒸馏机制,将复杂多步推理知识压缩为可部署的轻量决策网络。

利用上述多智能体系统在大规模合成退化数据集上离线运行,收集决策轨迹数据。每条轨迹记录输入图像、空间退化图、各区域最优工具序列与参数配置。基于该数据集训练轻量级路由网络,使其直接根据退化图像预测区域划分、退化类型及工具调度方案。

路由网络采用编码器—多任务解码器架构。编码器提取图像特征;解码器分为四个分支头,分别输出区域掩码、退化类型概率、工具选择概率和连续参数值。通过多任务联合训练,使网络同时学习分割、分类、路由与参数回归四项任务。

部署时仅需轻量路由网络与恢复工具库,无需依赖LLM/VLM在线推理,可实现端侧实时推理。为缓解蒸馏偏差问题,采用高质量轨迹重加权与困难样本再训练策略。

\subsection{重点难点}

本项目的重点难点包括:

1)\textbf{难点1(对应目标1)}:如何在零样本条件下准确分割退化区域并量化退化程度。退化导致的模糊边界、噪声纹理和对比度损失会干扰分割质量,造成过分割、分割漂移或边界不准确。拟解决路径:通过"语义分割 + 区域级退化推理 + 一致性约束"联合优化空间退化图质量;引入RobustSAM作为备选分割方案;设置区域置信度门控与候选掩码回退机制,降低分割误差向后续模块的传播。

2)\textbf{难点2(对应目标2)}:如何在空间维度上协调多智能体决策,避免区域边界不一致与策略冲突。不同区域专家的独立决策可能导致边界处处理痕迹明显,且工具链调用顺序会影响最终效果。拟解决路径:通过全局调度器的优先级约束实现跨区域协同;引入区域掩码渐变融合机制,在边界处实现平滑过渡;通过质量反馈闭环与全局一致性检查发现并修正区域间冲突。

3)\textbf{难点3(对应目标3)}:如何将复杂多步推理知识有效压缩到轻量网络中并保持性能稳定。多智能体的迭代优化策略涉及复杂的状态转移逻辑,轻量网络容量有限可能导致知识蒸馏不充分。拟解决路径:采用高质量轨迹筛选,仅使用多智能体收敛后的最优决策作为监督信号;引入困难样本重训练机制,针对蒸馏偏差较大的样本进行迭代优化;设置性能降级阈值,当蒸馏损失过大时回退到轻量网络+有限LLM调用的混合模式。

以上难点分别对应“感知—决策—部署”三类核心目标,上述解决路径已在技术路线中预留备选方案与回退机制,确保项目可执行性。
