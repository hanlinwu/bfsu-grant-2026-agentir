\section{前期准备工作}

\subsection{理论与技术储备}

(1)图像恢复与超分辨率方法学储备:系统梳理盲超分辨率重建、连续比例因子建模的理论框架,研究深度降质先验(Deep Degradation Prior)与在线学习机制在遥感影像复原中的应用,掌握跨尺度特征耦合与动态上采样方法,为区域工具参数连续优化提供理论基础。

(2)多模态遥感理解储备:深入分析视觉语言模型(VLM)在遥感场景理解中的适配策略,研究跨模态检索、变化描述与变化问答任务的技术路线,探索指令驱动视觉分析流程,为多智能体协同推理与反馈机制提供技术参考。

(3)生成式建模储备:研究标准化流(Normalizing Flow)、扩散模型(Diffusion Model)与隐式神经表示(INR)在复杂退化分布建模中的优势,分析生成式方法在真实场景退化恢复中的适用性,为空间退化图驱动的策略规划提供可迁移技术方案。

\subsection{数据与实验环境构建}

(1)遥感图像恢复实验环境搭建:配置基于PyTorch的统一训练与验证框架,集成主流超分辨率算法(如ESRGAN、Real-ESRGAN、SwinIR)的复现流程,建立标准化的数据预处理与评测指标计算模块,确保实验可稳定复现。

(2)多模态任务实验流程建设:构建跨模态任务的端到端实验管线,涵盖数据加载、特征提取、联合训练与结果可视化环节,形成可复用的数据处理与评测流程。

(3)多源遥感数据集整理:收集公开非配对遥感数据集(如Sentinel-2、Landsat、WorldView),构建低分辨率(LR)与高分辨率(HR)非配对样本库,涵盖不同传感器、光照条件与地物类型;开发数据预处理工具链(辐射校正、去云、配准),确保数据分布多样性。

\subsection{关键技术预研}

(1)连续比例因子恢复预研:针对效率与质量平衡问题,研究轻量化模型架构与动态参数策略,验证隐式神经表示在非整数倍率(如$\times$2.5、$\times$3.5)重建中的可行性,初步结果支持该技术路径的有效性。后续计划纳入区域级参数调优机制。

(2)指令驱动决策预研:针对复杂场景下的决策稳定性问题,研究视觉语言联合分析与反馈闭环机制,验证自然语言指令在遥感影像分析任务中的引导效果,初步确认指令驱动方法可有效提升分析准确性。后续计划扩展至多智能体交互流程。

(3)复杂退化建模预研:针对退化建模与部署效率的矛盾,研究生成式建模结合轨迹蒸馏的技术路线,验证潜空间建模在复杂退化场景下的鲁棒性,初步结果表明该方法可有效处理真实场景退化分布。后续计划推进轻量模型蒸馏与边缘部署测试。

(4)风险识别与改进策略:针对预研中可能出现的训练不稳定、跨场景泛化不足等问题,已制定困难样本增强、分阶段训练与多指标联合评估策略。

