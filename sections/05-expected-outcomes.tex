\section{预期成果}

本项目将构建多智能体协同的空间感知图像恢复系统,在理论与方法上形成“空间退化图表示—层级协同规划—轨迹蒸馏部署”的完整技术链。

预期成果与目标对应关系如下:

(1)对应目标1(感知):形成空间退化分析算法、评测脚本与实验数据资产,完成阶段论文投稿。

(2)对应目标2(决策):形成层级多智能体恢复原型系统与调度策略库,申请发明专利。

(3)对应目标3(部署):形成轻量路由部署模型与端侧测试报告,完成系统化验证。

总体成果计划为:在重要期刊或会议发表论文2--4篇(争取TIP、TPAMI、CVPR、ICCV等),申请发明专利1项,培养硕士研究生1--2名。

时间节点上,中期完成感知与协同模块验证,结题前完成蒸馏部署与综合评测。

